\chapter*{Abstract}
\addcontentsline{toc}{chapter}{Abstract}

This research investigates the integration of Environmental, Social, and Governance (ESG) criteria into credit risk modeling for bond portfolios, specifically examining the added value of Machine Learning techniques compared to traditional approaches. As institutional investors increasingly seek to balance financial performance with sustainability objectives, this study develops an innovative methodological framework to optimize this integration.

The central research question addresses how to most effectively incorporate ESG factors into credit risk assessment while precisely quantifying their contribution to model predictive capacity. The primary objective demonstrates that the combined use of ESG criteria and Machine Learning techniques can significantly improve credit risk model accuracy while simultaneously optimizing both financial and extra-financial portfolio performance.

The methodology encompasses three main components. First, we constructed an investment universe of 847 bonds selected using strict criteria for liquidity, credit quality, and ESG data availability, distributed across corporate bonds (60%), sovereign bonds (30%), and green or sustainable bonds (10%). Second, we implemented and compared nine Machine Learning models, ranging from traditional logistic regression to sophisticated ensemble approaches, using rigorous temporal cross-validation with a twelve-month prediction horizon. Third, we designed a multi-objective optimization problem for portfolio construction that explicitly integrates ESG constraints into the utility function.

The empirical results demonstrate significant superiority of Machine Learning models incorporating ESG factors. The stacking model, combining multiple base algorithms, achieves an AUC-ROC of 0.889, representing a 14.8-point improvement over traditional logistic regression (0.741). This superiority is particularly evident in anticipation capacity, with an Early Warning Score of 3.8 months compared to 1.9 months for classical approaches. ESG contribution analysis reveals that environmental, social, and governance factors collectively contribute 13.1% of total predictive performance, with governance criteria predominating (42% of ESG contribution), followed by environmental (35%) and social factors (23%).

Portfolio construction application confirms the practical value of this approach. The optimal portfolio achieves an ESG score of 74.1 points, representing a 12.9-point improvement over the benchmark, while maintaining a competitive Sharpe ratio of 0.89. This performance accompanies a substantial carbon footprint reduction (-35.9%) and doubled exposure to climate solutions, demonstrating the approach's capacity to contribute concretely to energy transition objectives.

Robustness tests reveal superior temporal stability of Machine Learning models under varying market conditions. The stacking model's Temporal Stability Ratio (2.68) indicates significantly higher resilience than traditional approaches, particularly valuable during economic stress periods. Climate transition scenarios confirm this robustness, with degradation limited to 8.9% in the worst case compared to 14.2% for logistic regression.

Comparative analysis between traditional and Machine Learning models reveals particularly pronounced performance gaps for High Yield issuers, where improvement reaches 11.3% in AUC-ROC compared to 7.6% for Investment Grade. This differentiation suggests that ESG-credit relationship complexity increasingly justifies advanced approaches as issuer risk profiles elevate.

The implications for bond management are multifaceted. Operational integration of advanced models requires a three-level decision architecture: strategic for asset allocation, tactical for issuer selection, and operational for execution and monitoring. This structure enables coherent risk signal incorporation across different temporal scales. Cost-benefit analysis indicates a profitability threshold of approximately 350-400 million euros in assets under management to justify investment in these sophisticated approaches.

This research contributes several original elements to academic literature and professional practices. Theoretically, it precisely quantifies ESG factors' contribution to credit risk prediction and demonstrates ensemble techniques' advantages for capturing complex interactions between financial and extra-financial variables. Methodologically, it proposes an operational multi-objective optimization framework for bond portfolio construction under ESG constraints. Empirically, it validates the hypothesis that responsible investment can create financial value through improved risk assessment.

Identified limitations primarily concern dependence on external ESG data, whose limited history and imperfect standardization may affect model robustness. Improvement prospects include integrating alternative data (satellite, textual), developing more sophisticated explainable models, and extending to other asset classes.

This research demonstrates that ESG criteria integration via advanced Machine Learning techniques constitutes a promising approach for simultaneously improving risk management and extra-financial impact of bond portfolios. The results suggest this methodological path can significantly contribute to finance's transformation toward a more sustainable model without compromising traditional performance objectives.

\textbf{Keywords:} Sustainable Finance, Credit Risk, Machine Learning, Bond Portfolio, ESG Criteria, Multi-objective Optimization, Ensemble Models, Risk Management