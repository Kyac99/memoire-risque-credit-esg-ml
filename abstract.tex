\chapter*{Abstract}
\addcontentsline{toc}{chapter}{Abstract}

This research examines the integration of Environmental, Social, and Governance (ESG) criteria into credit risk modeling for bond portfolios, specifically investigating the added value of Machine Learning techniques compared to traditional approaches.

The primary objective demonstrates that the combined use of ESG criteria and Machine Learning techniques can significantly improve credit risk model accuracy while simultaneously optimizing both financial and extra-financial portfolio performance.

The methodology encompasses three components. First, construction of an investment universe of 847 bonds distributed across corporate (60%), sovereign (30%), and green bonds (10%). Second, implementation and comparison of nine Machine Learning models using rigorous temporal cross-validation with a twelve-month prediction horizon. Third, design of a multi-objective optimization problem explicitly integrating ESG constraints into the utility function.

Results demonstrate significant superiority of Machine Learning models incorporating ESG factors. The stacking model achieves an AUC-ROC of 0.889, representing a 14.8-point improvement over traditional logistic regression (0.741). Early Warning Score reaches 3.8 months compared to 1.9 months for classical approaches. ESG factors collectively contribute 13.1% of total predictive performance, with governance criteria predominating (42%), followed by environmental (35%) and social factors (23%).

Portfolio construction application confirms practical value. The optimal portfolio achieves an ESG score of 74.1 points (+12.9 vs benchmark) while maintaining a competitive Sharpe ratio of 0.89. This performance accompanies a substantial carbon footprint reduction (-35.9%) and doubled exposure to climate solutions.

Robustness tests reveal superior temporal stability of advanced models, with the stacking model's Temporal Stability Ratio (2.68) indicating significantly higher resilience. Climate transition scenarios confirm robustness, with degradation limited to 8.9% in worst cases. Performance gaps are particularly pronounced for High Yield issuers (+11.3% in AUC-ROC).

Cost-benefit analysis indicates a profitability threshold of approximately 350-400 million euros in assets under management. Operational integration requires a three-level decision architecture covering strategic, tactical, and operational dimensions.

This research precisely quantifies ESG factors' contribution to credit risk prediction and validates that responsible investment can create financial value through improved risk assessment. Limitations primarily concern dependence on external ESG data and their imperfect standardization.

The research demonstrates that ESG criteria integration via advanced Machine Learning techniques constitutes a promising approach for simultaneously improving risk management and extra-financial impact of bond portfolios, contributing to finance transformation toward a more sustainable model without compromising traditional performance objectives.

\textbf{Keywords:} Sustainable Finance, Credit Risk, Machine Learning, Bond Portfolio, ESG Criteria, Multi-objective Optimization, Ensemble Models, Risk Management