\chapter{Bibliographie}

\begin{thebibliography}{100}

\bibitem{akouaydi2023}
Akouaydi, J., Abdallah, L., \& Ben Romdhane, L. (2023). A comparative study of machine learning algorithms for credit scoring. \textit{Proceedings of the International Conference on Machine Learning and Data Mining}, 150--165.

\bibitem{alaka2018}
Alaka, H. A., Oyedele, L. O., Owolabi, H. A., Kumar, V., Ajayi, S. O., Akinade, O. O., \& Bilal, M. (2018). A systematic review of bankruptcy prediction models: Towards a framework for tool selection. \textit{Expert Systems with Applications}, 94, 164--184.

\bibitem{altman1968}
Altman, E. I. (1968). Financial ratios, discriminant analysis and the prediction of corporate bankruptcy. \textit{The Journal of Finance}, 23(4), 589--609.

\bibitem{bai2021}
Bai, J., Bali, T. G., \& Wen, Q. (2021). Machine learning and corporate bond pricing. \textit{The Journal of Finance}, 76(6), 3331--3384.

\bibitem{bao2019}
Bao, Z., Denny, M., Iuga, I., \& Mishra, A. (2019). Multi-task learning for corporate sustainability performance. \textit{Finance Research Letters}, 30, 292--298.

\bibitem{barclays2018}
Barclays. (2018). \textit{The case for sustainable bond investing strengthens}. Barclays Investment Bank.

\bibitem{barth2022}
Barth, A., \& Wagner, A. F. (2022). How important is model risk when evaluating credit risk? \textit{Journal of Credit Risk}, 18(2), 1--30.

\bibitem{basel2017}
Basel Committee on Banking Supervision. (2017). Basel III: Finalising post-crisis reforms. \textit{Bank for International Settlements}.

\bibitem{bauer2018}
Bauer, D., Dimpfl, T., \& Fink, H. (2018). Data-driven network models for financial systemic risk. \textit{Journal of Economic Dynamics and Control}, 88, 212--232.

\bibitem{berg2022}
Berg, F., Kölbel, J. F., \& Rigobon, R. (2022). Aggregate confusion: The divergence of ESG ratings. \textit{Review of Finance}, 26(6), 1315--1344.

\bibitem{blanco2022}
Blanco, C., Rey-Marston, M., Piper, L., \& Monasterolo, I. (2022). Artificial intelligence and sustainable finance. \textit{AICPA \& CIMA}.

\bibitem{bolton2021}
Bolton, P., \& Kacperczyk, M. (2021). How green is your bond? Matching carbon footprints and bond returns. \textit{Journal of Financial Economics}.

\bibitem{borisov2022}
Borisov, V., Leemann, T., Seßler, K., Haug, J., Pawelczyk, M., \& Kasneci, G. (2022). Deep neural networks and tabular data: A survey. \textit{IEEE Transactions on Neural Networks and Learning Systems}.

\bibitem{breiman2001}
Breiman, L. (2001). Random forests. \textit{Machine Learning}, 45(1), 5--32.

\bibitem{brigo2006}
Brigo, D., \& Mercurio, F. (2006). \textit{Interest rate models-theory and practice: with smile, inflation and credit}. Springer Science \& Business Media.

\bibitem{bui2021}
Bui, M.-H., \& Huynh, T. L. D. (2021). Environmental uncertainty and ESG practices. \textit{Environmental Science and Pollution Research}, 28(36), 50997--51014.

\bibitem{burtis2018}
Burtis, B., Boysen, C., \& Rennard, D. (2018). A guide for companies: How to improve corporate ESG ratings. \textit{Journal of Environmental Investing}, 9(1).

\bibitem{campbell2008}
Campbell, J. Y., Hilscher, J., \& Szilagyi, J. (2008). In search of distress risk. \textit{The Journal of Finance}, 63(6), 2899--2939.

\bibitem{caplan2013}
Caplan, L., Griswold, J. S., \& Jarvis, W. F. (2013). Responsible investing: Principles, challenges, and opportunities. \textit{Commonfund Institute}.

\bibitem{capponi2020}
Capponi, A., Jia, R., \& Rios, D. X. (2020). Credit Risk Management Through the Lens of Machine Learning. \textit{Annual Review of Financial Economics}, 12, 341--365.

\bibitem{chava2022}
Chava, S., Humphery-Jenner, M., Kim, N.-E., Oh, S., \& Shekhar, S. (2022). Socially responsible investing versus vice investing. \textit{Journal of Corporate Finance}, 72, 102158.

\bibitem{chen2016}
Chen, T., \& Guestrin, C. (2016). Xgboost: A scalable tree boosting system. \textit{Proceedings of the 22nd ACM SIGKDD international conference on knowledge discovery and data mining}, 785--794.

\bibitem{chen2020}
Chen, X., Huang, J., \& Wang, F. (2020). ESG and credit risk: A new framework for systematic integration. \textit{The Journal of Fixed Income}, 29(4), 51--65.

\bibitem{cheng2021}
Cheng, I.-H., Hong, H., \& Shue, K. (2021). Why is green finance booming (not)? \textit{The Journal of Finance}, 76(4), 1635--1681.

\bibitem{clark2020}
Clark, G. L., Feiner, A., \& Viehs, M. (2020). ESG Integration in Investment Management: Myths and Realities. \textit{Journal of Business Ethics}, 165(2), 363--383.

\bibitem{cortes1995}
Cortes, C., \& Vapnik, V. (1995). Support-vector networks. \textit{Machine learning}, 20(3), 273--297.

\bibitem{demartini2015}
De Martini, G., \& Mainillo, G. (2015). A note on the relationship between ESG and credit ratings. \textit{Journal of Sustainable Finance \& Investment}, 5(4), 234--242.

\bibitem{desclee2016}
Desclée, A., Hyman, J., Dynkin, L., \& Polbennikov, S. (2016). Integrating ESG into fixed income. \textit{The Journal of Fixed Income}, 26(1), 5--20.

\bibitem{dixon2020}
Dixon, M. F., \& Halperin, I. (2020). Machine learning in finance: a topic modeling approach. \textit{European Journal of Finance}, 26(7-8), 731--751.

\bibitem{drempetic2020}
Drempetic, S., Klein, C., \& Zwergel, B. (2020). The influence of firm size on the ESG score: Corporate sustainability ratings under review. \textit{Journal of Business Ethics}, 167(2), 333--360.

\bibitem{duffie1999}
Duffie, D., \& Singleton, K. J. (1999). Modeling term structures of defaultable bonds. \textit{The Review of Financial Studies}, 12(4), 687--720.

\bibitem{dyduch2019}
Dyduch, J., \& Krasodomska, J. (2019). ESG factors in investment risk assessment. \textit{Journal of Business Ethics}, 156(3), 787--806.

\bibitem{eccles2014}
Eccles, R. G., Ioannou, I., \& Serafeim, G. (2014). The impact of corporate sustainability on organizational processes and performance. \textit{Management Science}, 60(11), 2835--2857.

\bibitem{efron1994}
Efron, B., \& Tibshirani, R. J. (1994). \textit{An introduction to the bootstrap}. CRC press.

\bibitem{ehsani2022}
Ehsani, S., \& Song, W. (2022). Reaching for yield or playing it safe? Risk taking by corporate bond funds. \textit{The Journal of Financial and Quantitative Analysis}, 57(6), 2260--2290.

\bibitem{eilers2022}
Eilers, F., Wachtel, K., \& Heinemann, K. (2022). ESG integration in credit risk: A performance assessment. \textit{Journal of Banking \& Finance}, 134, 106338.

\bibitem{fabozzi2012}
Fabozzi, F. J. (2012). \textit{The handbook of fixed income securities}. McGraw Hill Professional.

\bibitem{ferreira2019}
Ferreira, M. A., Matos, P., \& Portugal, R. (2019). ESG for all? The impact of ESG screening on return, risk, and diversification. \textit{Journal of Banking \& Finance}, 106, 114--130.

\bibitem{flammer2018}
Flammer, C. (2018). Corporate green bonds. \textit{Journal of Financial Economics}, 131(1), 226--247.

\bibitem{friede2015}
Friede, G., Busch, T., \& Bassen, A. (2015). ESG and financial performance: aggregated evidence from more than 2000 empirical studies. \textit{Journal of Sustainable Finance \& Investment}, 5(4), 210--233.

\bibitem{fuertes2020}
Fuertes, A., Soria, J. A., Muñoz-Torres, M., \& Ferrero-Ferrero, I. (2020). ESG data integration in credit risk assessment: Evidence from European banks. \textit{Sustainability}, 12(13), 5545.

\bibitem{giesecke2004}
Giesecke, K. (2004). \textit{Credit risk modeling and valuation: an introduction}. Now Publishers Inc.

\bibitem{giese2019}
Giese, G., Lee, L.-E., Melas, D., Nagy, Z., \& Nishikawa, L. (2019). Foundations of ESG investing: How ESG affects equity valuation, risk, and performance. \textit{The Journal of Portfolio Management}, 45(5), 69--83.

\bibitem{gompers2003}
Gompers, P., Ishii, J., \& Metrick, A. (2003). Corporate governance and equity prices. \textit{The Quarterly Journal of Economics}, 118(1), 107--156.

\bibitem{grinold2000}
Grinold, R. C., \& Kahn, R. N. (2000). \textit{Active portfolio management}. McGraw-Hill New York.

\bibitem{guotai2017}
Guotai, B., Bian, S., \& Coggins, R. (2017). Deep learning for corporate bond trading. \textit{Journal of Financial Data Science}, 1(2), 78--97.

\bibitem{hastie2009}
Hastie, T., Tibshirani, R., \& Friedman, J. H. (2009). \textit{The elements of statistical learning: data mining, inference, and prediction}. Springer.

\bibitem{henke2020}
Henke, H.-M. (2020). ESG and credit risk. \textit{Financial Analysts Journal}, 76(1), 53--65.

\bibitem{hochreiter1997}
Hochreiter, S., \& Schmidhuber, J. (1997). Long short-term memory. \textit{Neural Computation}, 9(8), 1735--1780.

\bibitem{hoepner2018}
Hoepner, A. G. F., Oikonomou, I., Sautner, Z., Starks, L. T., \& Zhou, X. (2018). ESG shareholder engagement and downside risk. \textit{Journal of Business Ethics}, 162, 167--191.

\bibitem{hong2019}
Hong, H., Li, F. W., \& Xu, J. (2019). Climate change, climate risk, and financial markets. \textit{Management Science}, 65(10), 4573--4599.

\bibitem{isda2019}
International Swaps and Derivatives Association. (2019). \textit{ISDA SIMM Methodology, version 2.2}. ISDA.

\bibitem{jarrow1995}
Jarrow, R. A., \& Turnbull, S. M. (1995). Pricing derivatives on financial securities subject to credit risk. \textit{The Journal of Finance}, 50(1), 53--85.

\bibitem{kashyap2018}
Kashyap, A. K. (2018). Machine learning for credit scoring: Improving logistic regression with non-linear decision-tree effects. \textit{SSRN Electronic Journal}.

\bibitem{ke2017}
Ke, G., Meng, Q., Finley, T., Wang, T., Chen, W., Ma, W., Ye, Q., \& Liu, T.-Y. (2017). LightGBM: A highly efficient gradient boosting decision tree. \textit{Advances in neural information processing systems}, 30.

\bibitem{khan2016}
Khan, M., Serafeim, G., \& Yoon, A. (2016). Corporate sustainability performance and bank loan pricing: It pays to be good, but only when banks are too. \textit{Working paper}.

\bibitem{kolbel2020}
Kölbel, J. F., Heeb, F., Paetzold, F., \& Busch, T. (2020). Can sustainable investing save the world? Reviewing the mechanisms of investor impact. \textit{Organization \& Environment}, 33(4), 554--574.

\bibitem{lepenioti2020}
Lepenioti, K., Bousdekis, A., Apostolou, D., \& Mentzas, G. (2020). Machine learning for credit scoring: A systematic literature review. \textit{IEEE Access}, 8, 220265--220294.

\bibitem{longstaff2005}
Longstaff, F. A., Mithal, S., \& Neis, E. (2005). Corporate yield spreads: Default risk or liquidity? New evidence from the credit default swap market. \textit{The Journal of Finance}, 60(5), 2213--2253.

\bibitem{lundberg2017}
Lundberg, S. M., \& Lee, S.-I. (2017). A unified approach to interpreting model predictions. \textit{Advances in neural information processing systems}, 30.

\bibitem{menz2010}
Menz, K.-M. (2010). Socially responsible investments: Performance and liquidity. \textit{The Journal of Banking and Finance}, 34(9), 2271--2280.

\bibitem{merton1974}
Merton, R. C. (1974). On the pricing of corporate debt: The risk structure of interest rates. \textit{The Journal of Finance}, 29(2), 449--470.

\bibitem{molnar2022}
Molnar, C., Casalicchio, G., \& Kästner, B. (2022). Interpretable machine learning: Fundamental principles and best practices. \textit{Information Fusion}, 84, 103--120.

\bibitem{moody2018}
Moody's Investors Service. (2018). \textit{Environmental, Social and Governance: How ESG risks influence credit ratings}. Moody's.

\bibitem{morningstar2020}
Morningstar Research. (2020). \textit{Sustainable Funds U.S. Landscape Report}. Morningstar.

\bibitem{msci2020}
MSCI. (2020). \textit{ESG Ratings Methodology}. MSCI ESG Research LLC.

\bibitem{pastor2022}
Pastor, L., Stambaugh, R. F., \& Taylor, L. A. (2022). Sustainable investing in equilibrium. \textit{Journal of Financial Economics}, 143(2), 521--549.

\bibitem{pedregosa2011}
Pedregosa, F., Varoquaux, G., Gramfort, A., Michel, V., Thirion, B., Grisel, O., Blondel, M., Prettenhofer, P., Weiss, R., Dubourg, V., \& others. (2011). \textit{Scikit-learn: Machine learning in Python}. Journal of Machine Learning Research.

\bibitem{polbennikov2016}
Polbennikov, S., Desclée, A., Dynkin, L., \& Maitra, A. (2016). ESG ratings and performance of corporate bonds. \textit{The Journal of Fixed Income}, 26(1), 21--41.

\bibitem{prokhorenkova2018}
Prokhorenkova, L., Gusev, G., Vorobev, A., Dorogush, A. V., \& Gulin, A. (2018). CatBoost: unbiased boosting with categorical features. \textit{Advances in neural information processing systems}, 31.

\bibitem{rebonato2018}
Rebonato, R. (2018). \textit{Bond Pricing and Yield Curve Modeling: A Structural Approach}. Cambridge University Press.

\bibitem{ribeiro2016}
Ribeiro, M. T., Singh, S., \& Guestrin, C. (2016). Why should I trust you? Explaining the predictions of any classifier. \textit{Proceedings of the 22nd ACM SIGKDD international conference on knowledge discovery and data mining}, 1135--1144.

\bibitem{roncalli2020}
Roncalli, T. (2020). \textit{Handbook of Financial Risk Management: Asset Liability Management and Interest Rate Risk}. Chapman and Hall/CRC.

\bibitem{rudin2019}
Rudin, C. (2019). Stop explaining black box machine learning models for high stakes decisions and use interpretable models instead. \textit{Nature Machine Intelligence}, 1(5), 206--215.

\bibitem{serafeim2020}
Serafeim, G. (2020). Social-impact efforts that create real value. \textit{Harvard Business Review}, 98(5), 38--48.

\bibitem{shumway2001}
Shumway, T. (2001). Forecasting bankruptcy more accurately: A simple hazard model. \textit{The Journal of Business}, 74(1), 101--124.

\bibitem{tcfd2017}
Task Force on Climate-related Financial Disclosures. (2017). \textit{Recommendations of the Task Force on Climate-related Financial Disclosures}. Financial Stability Board.

\bibitem{tsay2005}
Tsay, R. S. (2005). \textit{Analysis of financial time series}. John Wiley \& Sons.

\bibitem{weber2014}
Weber, O., Mansfeld, M., \& Schirrmann, E. (2014). The financial performance of sustainability indexes: An empirical study of the relationship between ESG rating and market performance. \textit{Journal of Sustainable Finance \& Investment}, 4(3), 227--243.

\bibitem{xing2020}
Xing, F. Z., Cambria, E., \& Welsch, R. E. (2020). Natural language based financial forecasting: a survey. \textit{Artificial Intelligence Review}, 53(2), 795--823.

\bibitem{zerbib2019}
Zerbib, O. D. (2019). The effect of pro-environmental preferences on bond prices: Evidence from green bonds. \textit{Journal of Banking \& Finance}, 98, 39--60.

\bibitem{zopounidis2002}
Zopounidis, C., \& Doumpos, M. (2002). Multi-criteria decision aid in financial decision making: methodologies and literature review. \textit{Journal of Multi-Criteria Decision Analysis}, 11(4-5), 167--186.

\end{thebibliography}