\chapter{Comparaison entre modèles traditionnels et Machine Learning}

\section{Analyse comparative des performances}

La comparaison rigoureuse entre approches traditionnelles et modèles de Machine Learning constitue une étape essentielle pour évaluer la valeur ajoutée de ces derniers dans la modélisation du risque de crédit intégrant les facteurs ESG. Cette section présente une analyse détaillée des performances relatives selon différentes dimensions.

\subsection{Précision des prédictions : analyse comparative globale}

Pour établir une comparaison équitable, nous avons implémenté et évalué sur le même ensemble de test (juillet 2022 - décembre 2023) les modèles suivants :

\subsubsection{Modèles traditionnels}
\begin{itemize}
  \item Modèle structurel de Merton modifié (KMV)
  \item Modèle à forme réduite (Jarrow-Turnbull)
  \item Modèle Z-score d'Altman adapté
  \item Régression logistique multivariée
  \item Modèle de scoring interne bancaire (approche standard)
\end{itemize}

\subsubsection{Modèles de Machine Learning}
\begin{itemize}
  \item XGBoost (meilleur modèle individuel)
  \item Ensemble modulaire ESG-Financier (approche modulaire)
  \item Stacking (meilleure performance globale)
\end{itemize}

Le tableau comparatif complet des performances entre modèles traditionnels et de Machine Learning se présente comme suit :

\begin{table}[htbp]
  \centering
  \caption{Comparaison des performances entre modèles traditionnels et Machine Learning}
  \begin{tabular}{llccccc}
    \toprule
    \textbf{Catégorie} & \textbf{Modèle} & \textbf{AUC-ROC} & \textbf{Précision} & \textbf{Rappel} & \textbf{F1-Score} & \textbf{Log-Loss} \\
    \midrule
    Traditionnels & Merton (KMV) & 0,783 & 0,726 & 0,682 & 0,703 & 0,562 \\
    Traditionnels & Forme réduite & 0,792 & 0,735 & 0,687 & 0,710 & 0,551 \\
    Traditionnels & Z-score adapté & 0,768 & 0,719 & 0,671 & 0,694 & 0,573 \\
    Traditionnels & Régression logistique & 0,804 & 0,753 & 0,698 & 0,724 & 0,521 \\
    Traditionnels & Scoring bancaire & 0,794 & 0,741 & 0,692 & 0,716 & 0,547 \\
    Machine Learning & XGBoost & 0,881 & 0,815 & 0,763 & 0,788 & 0,412 \\
    Machine Learning & Ensemble modulaire & 0,884 & 0,819 & 0,767 & 0,792 & 0,408 \\
    Machine Learning & Stacking & \textbf{0,893} & \textbf{0,827} & \textbf{0,775} & \textbf{0,800} & \textbf{0,394} \\
    \bottomrule
  \end{tabular}
\end{table}

\subsubsection{Analyse des écarts de performance}

L'analyse détaillée de ces résultats révèle plusieurs tendances significatives :

\begin{enumerate}
  \item \textbf{Gain de performance substantiel} : Les modèles de Machine Learning surpassent systématiquement les approches traditionnelles avec un gain moyen de 9,7 points de pourcentage en AUC-ROC (0,886 contre 0,789).

  \item \textbf{Amélioration du rappel} : L'écart est particulièrement marqué pour le rappel (+7,8 points en moyenne), indiquant une meilleure capacité des modèles ML à identifier les cas de détérioration du crédit, un avantage crucial pour la gestion des risques.

  \item \textbf{Réduction de l'erreur logarithmique} : La diminution moyenne de 28,5\% du log-loss témoigne d'une meilleure calibration des probabilités prédites par les modèles ML, garantissant des estimations de risque plus fiables.

  \item \textbf{Gradient de complexité} : On observe une corrélation positive entre la complexité des modèles et leurs performances, le stacking obtenant les meilleurs résultats en combinant les forces de plusieurs approches.
\end{enumerate}

\subsection{Performance par classe de notation}

L'analyse des performances segmentées par catégorie de notation révèle des nuances importantes :

\begin{table}[htbp]
  \centering
  \caption{Performance par catégorie de notation}
  \begin{tabular}{lccc}
    \toprule
    \textbf{Catégorie de notation} & \textbf{AUC-ROC Modèles traditionnels} & \textbf{AUC-ROC Machine Learning} & \textbf{Différence} \\
    \midrule
    Investment Grade (AAA-BBB-) & 0,765 & 0,823 & +7,6\% \\
    High Yield supérieur (BB+/BB/BB-) & 0,803 & 0,894 & +11,3\% \\
    High Yield inférieur (B+/B/B-) & 0,827 & 0,921 & +11,4\% \\
    Très spéculatif (CCC+/CCC/CCC-) & 0,812 & 0,903 & +11,2\% \\
    \bottomrule
  \end{tabular}
\end{table}

Cette segmentation met en évidence une supériorité plus marquée des modèles de Machine Learning pour les émetteurs spéculatifs (High Yield), où la complexité et la non-linéarité des relations entre variables explicatives et risque de crédit sont plus prononcées. À l'inverse, l'écart est moins important pour les émetteurs Investment Grade, aux caractéristiques plus stables et aux relations risque-rendement plus linéaires.

\subsection{Robustesse face aux variations de marché}

La capacité des modèles à maintenir leurs performances prédictives dans différentes conditions de marché constitue un critère d'évaluation essentiel. Pour tester cette robustesse, nous avons analysé les performances sur trois sous-périodes distinctes de l'ensemble de test :

\begin{enumerate}
  \item \textbf{Période de stabilité relative} (juillet 2022 - octobre 2022)
  \item \textbf{Période de stress modéré} (novembre 2022 - mars 2023)
  \item \textbf{Période volatile} (avril 2023 - décembre 2023)
\end{enumerate}

Les résultats d'AUC-ROC par période sont présentés ci-dessous :

\begin{table}[htbp]
  \centering
  \caption{Performance des modèles par période de marché}
  \begin{tabular}{llcccc}
    \toprule
    \textbf{Catégorie} & \textbf{Modèle} & \textbf{Période stable} & \textbf{Période stress modéré} & \textbf{Période volatile} & \textbf{Écart-type} \\
    \midrule
    Traditionnels & Merton (KMV) & 0,802 & 0,776 & 0,761 & 0,021 \\
    Traditionnels & Forme réduite & 0,814 & 0,782 & 0,772 & 0,022 \\
    Traditionnels & Z-score adapté & 0,785 & 0,763 & 0,739 & 0,023 \\
    Traditionnels & Régression logistique & 0,823 & 0,798 & 0,783 & 0,020 \\
    Traditionnels & Scoring bancaire & 0,817 & 0,789 & 0,769 & 0,024 \\
    Machine Learning & XGBoost & 0,894 & 0,876 & 0,869 & 0,013 \\
    Machine Learning & Ensemble modulaire & 0,897 & 0,881 & 0,872 & 0,013 \\
    Machine Learning & Stacking & \textbf{0,905} & \textbf{0,889} & \textbf{0,878} & \textbf{0,014} \\
    \bottomrule
  \end{tabular}
\end{table}

Ces résultats mettent en évidence plusieurs caractéristiques importantes :

\begin{enumerate}
  \item \textbf{Dégradation plus limitée en période volatile} : Les modèles de Machine Learning présentent une baisse de performance moins prononcée en période de stress (-3,0\% en moyenne contre -5,3\% pour les modèles traditionnels).

  \item \textbf{Meilleure stabilité} : L'écart-type des performances entre périodes est significativement plus faible pour les approches ML (0,013 contre 0,022 en moyenne), témoignant d'une plus grande stabilité temporelle.

  \item \textbf{Résilience du stacking} : L'approche par ensemble maintient les meilleures performances dans toutes les conditions de marché, confirmant l'intérêt de la diversification des modèles pour améliorer la robustesse.

  \item \textbf{Apport des facteurs ESG} : Une analyse supplémentaire montre que l'intégration des variables ESG contribue significativement à cette stabilité, avec une réduction de 31\% de la dégradation des performances en période volatile par rapport aux modèles n'utilisant que des facteurs financiers traditionnels.
\end{enumerate}

Cette capacité supérieure des modèles de Machine Learning à maintenir leurs performances prédictives en conditions de marché difficiles constitue un avantage considérable pour la gestion des risques, permettant une anticipation plus fiable des détériorations de crédit dans les périodes où cette capacité est la plus précieuse.

\section{Avantages et limites des approches}

Les résultats comparatifs présentés précédemment mettent en évidence des différences fondamentales entre modèles traditionnels et approches par Machine Learning. Cette section analyse en profondeur les avantages et limites respectifs de ces deux familles d'approches.

\subsection{Machine Learning : avantages et limitations}

\subsubsection{Avantages des approches par Machine Learning}

\begin{enumerate}
  \item \textbf{Capture des relations non-linéaires et interactions complexes} : \\
  Les algorithmes de ML, particulièrement les modèles d'ensemble et les réseaux neuronaux, excellent dans l'identification de relations non-linéaires entre variables explicatives et risque de crédit. L'analyse des graphiques de dépendance partielle (PDP) révèle des effets de seuil, des plateaux et des inflexions que les modèles linéaires ne peuvent capturer. Cette capacité est particulièrement précieuse pour modéliser l'impact des facteurs ESG, souvent caractérisés par des effets non-proportionnels (par exemple, l'effet disproportionné des controverses graves sur le risque de crédit).

  \item \textbf{Traitement efficace de données hétérogènes et volumineuses} : \\
  Les modèles ML peuvent intégrer simultanément des centaines de variables de différentes natures (numériques, catégorielles, temporelles) sans nécessiter de spécification a priori des relations fonctionnelles. Cette flexibilité permet d'exploiter pleinement la richesse des données financières et extra-financières disponibles, y compris les données non structurées.

  \item \textbf{Adaptation dynamique} : \\
  La capacité d'apprentissage continu des modèles ML permet une adaptation plus rapide aux évolutions du contexte économique et réglementaire. Nos expérimentations montrent qu'après réentraînement sur des données récentes, les modèles ML retrouvent des performances optimales en 2-3 mois, contre 6-9 mois pour les modèles paramétriques traditionnels.

  \item \textbf{Meilleure performance prédictive globale} : \\
  Comme démontré dans la section précédente, les modèles ML offrent un gain substantiel en termes de précision et de rappel, particulièrement précieux pour anticiper les dégradations de crédit avant qu'elles ne soient reflétées dans les spreads ou les notations.

  \item \textbf{Capacité à intégrer nativement les facteurs ESG} : \\
  Les modèles ML déterminent automatiquement la pondération optimale des variables ESG en fonction de leur pouvoir prédictif, sans nécessiter d'hypothèses préalables sur leur importance relative. Cette approche data-driven permet une intégration plus organique et évolutive des critères ESG.
\end{enumerate}

\subsubsection{Limitations des approches par Machine Learning}

\begin{enumerate}
  \item \textbf{Défi d'interprétabilité ("boîte noire")} : \\
  Malgré les avancées récentes des techniques d'interprétabilité (SHAP, LIME), les modèles ML complexes restent moins transparents que les approches paramétriques traditionnelles. Cette opacité relative peut constituer un frein à leur adoption, particulièrement dans un contexte réglementaire exigeant en matière de transparence.

  \item \textbf{Risque de surapprentissage} : \\
  Les modèles ML sophistiqués peuvent capturer des patterns spécifiques à l'échantillon d'entraînement sans valeur prédictive réelle. Bien que les techniques de régularisation et de validation croisée atténuent ce risque, il demeure une préoccupation, surtout en finance où les régimes de marché évoluent.

  \item \textbf{Dépendance aux données historiques} : \\
  L'efficacité des modèles ML repose sur la disponibilité de données historiques représentatives. Or, dans le domaine ESG, l'historique de données standardisées reste limité (généralement post-2015), ce qui peut restreindre la capacité des modèles à capturer les dynamiques sur cycle économique complet.

  \item \textbf{Exigences computationnelles} : \\
  L'entraînement et l'optimisation des modèles ML avancés nécessitent des ressources computationnelles significatives, particulièrement pour les approches d'ensemble et les réseaux neuronaux profonds. Ces exigences peuvent représenter un obstacle opérationnel pour certaines organisations.

  \item \textbf{Validation réglementaire} : \\
  L'adoption des modèles ML dans les cadres réglementaires formels (Bâle, IFRS 9) reste limitée, nécessitant souvent des justifications supplémentaires et des procédures de validation spécifiques qui peuvent ralentir leur déploiement opérationnel.
\end{enumerate}

\subsection{Modèles traditionnels : forces et faiblesses}

\subsubsection{Forces des approches traditionnelles}

\begin{enumerate}
  \item \textbf{Transparence et interprétabilité intrinsèque} : \\
  Les modèles paramétriques classiques (régression logistique, Z-score) offrent une transparence native, chaque coefficient pouvant être directement interprété en termes d'impact marginal sur le risque. Cette clarté facilite la communication avec les parties prenantes non-techniques et répond aux exigences réglementaires.

  \item \textbf{Fondements théoriques solides} : \\
  Les modèles comme Merton ou Jarrow-Turnbull s'appuient sur des théories financières établies, ce qui renforce leur crédibilité et facilite leur validation conceptuelle. Ces fondements théoriques permettent également de mieux anticiper leur comportement dans des situations extrêmes ou inédites.

  \item \textbf{Parcimonie et robustesse} : \\
  Le nombre limité de paramètres des modèles traditionnels (typiquement 5-15 variables) réduit le risque de surapprentissage et peut offrir une meilleure robustesse face à des changements structurels du marché non observés dans les données historiques.

  \item \textbf{Acceptation réglementaire et institutionnelle} : \\
  Les approches traditionnelles bénéficient d'une large reconnaissance dans les cadres réglementaires et les pratiques institutionnelles, facilitant leur déploiement et leur validation. Cette acceptation représente un avantage pratique non négligeable.

  \item \textbf{Moindres exigences en données} : \\
  Les modèles classiques peuvent être implémentés avec des jeux de données plus restreints, ce qui présente un avantage pour l'analyse d'émetteurs moins couverts ou de marchés émergents où les données ESG détaillées peuvent être limitées.
\end{enumerate}

\subsubsection{Faiblesses des approches traditionnelles}

\begin{enumerate}
  \item \textbf{Hypothèse de linéarité restrictive} : \\
  La plupart des modèles économétriques classiques supposent des relations linéaires ou log-linéaires, une simplification qui limite leur capacité à capturer les dynamiques complexes du risque de crédit, particulièrement en période de stress.

  \item \textbf{Flexibilité limitée face à de nouvelles variables} : \\
  L'intégration de nouveaux facteurs explicatifs, comme les variables ESG, nécessite souvent une restructuration significative des modèles traditionnels, rendant l'évolution de ces derniers plus laborieuse.

  \item \textbf{Moindre granularité des prédictions} : \\
  Les approches classiques tendent à produire des distributions de probabilités moins nuancées, avec une concentration excessive autour de valeurs moyennes. Cette limitation réduit leur capacité à identifier les cas extrêmes.

  \item \textbf{Faible capacité à exploiter les données non structurées} : \\
  Les modèles traditionnels sont mal équipés pour intégrer directement des sources d'information qualitatives ou non structurées (rapports ESG narratifs, actualités) pourtant riches en signaux pertinents.

  \item \textbf{Performance prédictive inférieure} : \\
  Comme démontré dans notre analyse comparative, les modèles traditionnels présentent systématiquement des performances inférieures, particulièrement pour les émetteurs à profil de risque complexe où les interactions entre facteurs financiers et ESG sont déterminantes.
\end{enumerate}

\subsection{Analyse des compromis et complémentarités}

L'opposition entre modèles traditionnels et Machine Learning ne doit pas être perçue comme binaire. Notre analyse suggère plutôt une complémentarité potentielle, avec différents modèles présentant des avantages distincts selon le contexte d'utilisation :

\begin{table}[htbp]
  \centering
  \caption{Recommandations selon le contexte d'utilisation}
  \begin{tabular}{lll}
    \toprule
    \textbf{Contexte d'utilisation} & \textbf{Approche recommandée} & \textbf{Justification} \\
    \midrule
    Évaluation standardisée Investment Grade & Modèle traditionnel amélioré & Transparence supérieure, relations plus linéaires, acceptation réglementaire \\
    Détection précoce de détérioration & Modèle ML (XGBoost/LSTM) & Meilleur rappel, capacité à détecter des signaux faibles \\
    Émetteurs High Yield complexes & Ensemble ML avec facteurs ESG & Capture des non-linéarités, intégration efficace des risques extra-financiers \\
    Émergents/données limitées & Modèle traditionnel + variables ESG sélectives & Robustesse avec données limitées \\
    Stress testing réglementaire & Approche hybride avec dominante traditionnelle & Conformité réglementaire avec amélioration ML ciblée \\
    Gestion dynamique de portefeuille & Stacking ML avec composante temporelle & Performance optimale, adaptation rapide aux changements de marché \\
    \bottomrule
  \end{tabular}
\end{table}

Cette analyse nuancée suggère qu'une approche hybride, capitalisant sur les forces complémentaires des différentes méthodologies, peut offrir la solution la plus équilibrée pour l'évaluation du risque de crédit intégrant les facteurs ESG.

\section{Implications pour la gestion d'un portefeuille obligataire}

L'intégration des modèles avancés de risque de crédit, en particulier ceux incorporant les facteurs ESG via des techniques de Machine Learning, transforme potentiellement les pratiques de gestion obligataire. Cette section explore les implications concrètes pour les différentes dimensions de la gestion de portefeuille.

\subsection{Intégration opérationnelle des modèles dans la prise de décision}

L'implémentation efficace des modèles avancés dans les processus d'investissement nécessite une architecture décisionnelle structurée :

\subsubsection{Cadre d'intégration à trois niveaux}

\paragraph{Niveau stratégique (allocation d'actifs)}
\begin{itemize}
  \item Utilisation des prédictions agrégées pour ajuster l'exposition sectorielle et la pondération entre Investment Grade et High Yield
  \item Intégration des indicateurs de risque systémique ESG (comme l'exposition globale aux risques climatiques) dans la construction de scénarios macro
  \item Calibration des limites d'exposition en fonction des signaux de risque issus des modèles
\end{itemize}

\paragraph{Niveau tactique (sélection d'émetteurs)}
\begin{itemize}
  \item Mise en place d'un système de scoring multi-modèle combinant approches traditionnelles et ML
  \item Développement d'alertes précoces basées sur les variations de probabilité de dégradation
  \item Ajustement dynamique des primes de risque exigées en fonction des prédictions de risque idiosyncratique
\end{itemize}

\paragraph{Niveau opérationnel (exécution et suivi)}
\begin{itemize}
  \item Automatisation du monitoring des facteurs de risque identifiés comme matériels par les modèles
  \item Intégration des prédictions dans les systèmes de trading algorithmique pour l'optimisation des prix d'entrée/sortie
  \item Production automatisée de rapports de risque incorporant les métriques traditionnelles et les signaux ML
\end{itemize}

\subsubsection{Gouvernance et validation}

L'adoption des modèles avancés nécessite un cadre de gouvernance spécifique :

\begin{itemize}
  \item \textbf{Comité de validation des modèles} intégrant experts financiers, data scientists et spécialistes ESG
  \item \textbf{Processus de challenge} systématique des prédictions extrêmes ou contre-intuitives
  \item \textbf{Tests de back-testing} réguliers avec différentes métriques d'évaluation
  \item \textbf{Simulations de stress} spécifiques aux facteurs ESG identifiés comme matériels
  \item \textbf{Documentation standardisée} des hypothèses et limites des modèles
\end{itemize}

\subsubsection{Considérations opérationnelles}

\begin{itemize}
  \item Développement d'\textbf{interfaces utilisateur intuitives} permettant aux gestionnaires de comprendre les prédictions
  \item Mise en place de \textbf{pipelines de données automatisés} pour l'actualisation régulière des modèles
  \item Définition de \textbf{seuils d'intervention} calibrés selon le profil de risque du portefeuille
  \item \textbf{Formation des équipes} à l'interprétation et l'utilisation appropriée des signaux des modèles
\end{itemize}

\subsection{Perspectives pour la gestion des risques et l'investissement responsable}

L'intégration des modèles avancés ouvre de nouvelles perspectives pour concilier performance financière et objectifs d'investissement responsable :

\subsubsection{Innovation en gestion des risques}

\begin{enumerate}
  \item \textbf{Modélisation dynamique des corrélations} :\\
  Nos modèles ML suggèrent que les corrélations entre facteurs ESG et risque de crédit varient considérablement selon les régimes de marché. Cette observation permet d'envisager des stratégies de couverture dynamiques adaptées au contexte macroéconomique et à l'intensité des préoccupations ESG.

  \item \textbf{Analyse de scénarios ESG spécifiques} :\\
  Les capacités prédictives des modèles ML permettent de simuler l'impact d'événements ESG spécifiques (nouvelles réglementations climatiques, controverses sociales) sur le profil de risque du portefeuille, affinant ainsi les exercices de stress testing traditionnels.

  \item \textbf{Mesures de risque conditionnelles} :\\
  Développement de métriques de risque conditionnées par des facteurs ESG, comme la "Value-at-Risk sous stress climatique" ou le "spread ajusté au risque de transition énergétique".
\end{enumerate}

\subsubsection{Transformation de l'investissement responsable}

\begin{enumerate}
  \item \textbf{Optimisation multicritère avancée} :\\
  Les techniques de ML permettent de construire des frontières efficientes intégrant simultanément rendement financier, risque et impact ESG, dépassant les approches d'exclusion ou de best-in-class traditionnelles.

  \item \textbf{Stratification ESG fine} :\\
  Identification de "poches" d'émetteurs présentant des caractéristiques ESG distinctives mais homogènes, permettant une diversification plus sophistiquée que les approches sectorielles classiques.

  \item \textbf{Monitoring d'impact en temps réel} :\\
  Développement d'indicateurs dynamiques mesurant l'empreinte environnementale et sociale du portefeuille, calibrés sur les facteurs identifiés comme matériels par les modèles.
\end{enumerate}

\subsubsection{Tendances émergentes et opportunités}

\begin{enumerate}
  \item \textbf{Obligations climatiques structurées} :\\
  Conception de produits obligataires dont les caractéristiques (coupon, maturité) s'ajustent en fonction de l'atteinte d'objectifs ESG mesurables, avec pricing facilité par les modèles ML.

  \item \textbf{Stratégies de crédit thématiques} :\\
  Développement de stratégies ciblant spécifiquement les émetteurs bien positionnés face aux transitions (énergétique, numérique, sociétale) identifiées par les modèles comme porteuses de valeur à long terme.

  \item \textbf{Arbitrage d'inefficiences ESG} :\\
  Identification systématique d'émetteurs mal évalués par le marché en raison d'une appréciation incorrecte de leur profil ESG, offrant des opportunités d'alpha.
\end{enumerate}

\subsection{Intégration dans un cycle d'investissement complet}

Pour maximiser la valeur ajoutée des modèles avancés, leur intégration doit s'inscrire dans un cycle d'investissement complet :

\subsubsection{Processus d'investissement intégré}

\paragraph{Analyse fondamentale enrichie}
\begin{itemize}
  \item Intégration des facteurs identifiés comme matériels par les modèles ML
  \item Focus analytique guidé par l'importance des variables révélée par l'analyse SHAP
  \item Complémentarité entre jugement d'analyste et signaux quantitatifs
\end{itemize}

\paragraph{Construction de portefeuille optimisée}
\begin{itemize}
  \item Utilisation des probabilités de transition conditionnelles pour l'optimisation du portefeuille
  \item Allocation tenant compte des corrélations conditionnelles révélées par les modèles
  \item Diversification active guidée par les clusters de risque identifiés par apprentissage non supervisé
\end{itemize}

\paragraph{Exécution informée}
\begin{itemize}
  \item Timing des transactions guidé par les signaux de détérioration/amélioration précoces
  \item Détermination des tailles de position en fonction de la confiance des prédictions
  \item Gestion dynamique de la liquidité basée sur les anticipations de stress
\end{itemize}

\paragraph{Monitoring et ajustement continu}
\begin{itemize}
  \item Suivi en temps réel des indicateurs avancés identifiés par les modèles
  \item Mécanismes d'alerte basés sur les déviations significatives des variables clés
  \item Rebalancement guidé par les évolutions des profils de risque prédits
\end{itemize}

L'application systématique de ce cadre intégré peut transformer l'approche traditionnelle de la gestion obligataire, en permettant une anticipation plus fine des évolutions de crédit et une meilleure valorisation des facteurs ESG matériels, conduisant potentiellement à une amélioration simultanée du rendement ajusté au risque et de l'impact extra-financier du portefeuille.