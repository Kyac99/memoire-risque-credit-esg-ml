\chapter{Introduction}

\section{Contexte et enjeux}

Le risque de crédit représente l'un des principaux risques financiers auxquels sont confrontés les investisseurs en obligations. La crise financière de 2008 a souligné l'importance cruciale d'une évaluation rigoureuse de ce risque, particulièrement dans les portefeuilles obligataires où sa gestion détermine la performance et la stabilité à long terme.

Parallèlement, l'intégration des critères Environnementaux, Sociaux et de Gouvernance (ESG) a transformé le paysage de la gestion d'actifs. Les investisseurs incorporent désormais ces dimensions pour évaluer non seulement la rentabilité financière mais également l'impact sociétal et environnemental. Cependant, leur intégration systématique et quantifiable dans les modèles de risque de crédit demeure limitée et constitue un défi méthodologique majeur.

Les avancées en Machine Learning offrent de nouvelles perspectives pour la modélisation du risque de crédit. Contrairement aux modèles traditionnels basés sur des hypothèses simplificatrices et des relations linéaires, les techniques d'apprentissage automatique permettent de capturer des interactions complexes et non linéaires entre multiples variables, incluant les facteurs ESG.

\section{Problématique}

L'évaluation du risque de crédit s'appuie historiquement sur des modèles traditionnels (Merton, Jarrow & Turnbull, approches statistiques) qui présentent des limites significatives : dépendance à des hypothèses simplificatrices, conception statique de la dynamique des prix, et incapacité à intégrer efficacement des facteurs extra-financiers.

L'émergence des critères ESG incite investisseurs et régulateurs à intégrer ces dimensions dans l'analyse du risque. Des recherches suggèrent que les entreprises adoptant de solides pratiques ESG affichent un risque de défaut plus faible, mais la formalisation et la quantification précises de cet impact demeurent insuffisamment développées.

Les techniques de Machine Learning permettent d'identifier des relations complexes entre multiples variables explicatives, incluant des données ESG, financières et macroéconomiques. Leur application à la gestion obligataire soulève plusieurs questions : surpassent-elles les approches traditionnelles en précision et robustesse ? Comment garantir leur interprétabilité ? Dans quelle mesure permettent-elles une intégration plus efficace des critères ESG ?

Ce mémoire examine la question centrale suivante :

\textbf{Comment intégrer efficacement les critères ESG dans la modélisation du risque de crédit d'un portefeuille obligataire, et dans quelle mesure les modèles de Machine Learning permettent-ils d'améliorer la précision des prévisions par rapport aux modèles traditionnels ?}

\section{Objectifs et méthodologie}

Ce travail vise trois objectifs principaux : analyser l'impact des critères ESG sur le risque de crédit, comparer les approches traditionnelles et de Machine Learning, et développer une méthodologie intégrée combinant Machine Learning et critères ESG.

La méthodologie s'articule autour de la constitution d'un jeu de données combinant informations financières, données ESG et variables macroéconomiques, l'implémentation de modèles traditionnels et d'algorithmes de Machine Learning, et l'évaluation comparative rigoureuse selon des métriques objectives.

\section{Hypothèses de recherche}

Cinq hypothèses guident ce travail empirique : l'intégration ESG améliore significativement le pouvoir prédictif, la dimension Gouvernance a l'impact le plus significatif, les modèles de Machine Learning surpassent les approches traditionnelles particulièrement avec les variables ESG, leur avantage est marqué dans la capture des relations non linéaires, et la matérialité ESG varie selon les secteurs d'activité.

\section{Structure du mémoire}

Ce mémoire comprend cinq chapitres suivant une progression logique. Le Chapitre 1 présente la revue de littérature couvrant modèles traditionnels, impact ESG et Machine Learning financier. Le Chapitre 2 décrit le portefeuille obligataire étudié. Le Chapitre 3 aborde la modélisation traditionnelle du risque de crédit et l'intégration ESG. Le Chapitre 4 se concentre sur l'application des modèles de Machine Learning. Le Chapitre 5 propose une comparaison systématique entre approches traditionnelles et techniques avancées, analysant leurs performances relatives et implications pour la gestion de portefeuille.