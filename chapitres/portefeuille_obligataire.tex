\\section{Méthodologie quantitative d'optimisation de portefeuille avec contraintes ESG}\n\n\\subsection{Cadre théorique d'optimisation sous contraintes ESG}\n\nL'intégration des critères ESG dans la construction d'un portefeuille obligataire nécessite une adaptation des cadres d'optimisation traditionnels pour incorporer explicitement ces considérations extra-financières. Cette section présente le cadre théorique et mathématique sous-tendant notre approche d'optimisation.\n\n\\subsubsection{Formulation du problème d'optimisation multi-objectif}\n\nLa construction d'un portefeuille obligataire intégrant les critères ESG peut être formulée comme un problème d'optimisation multi-objectif, cherchant à équilibrer performance financière et performance extra-financière. Formellement, nous cherchons à résoudre :\n\n\\begin{align}\n\\max_{\\mathbf{w}} \\quad & U(\\mathbf{w}) = (1-\\lambda) \\cdot U_{fin}(\\mathbf{w}) + \\lambda \\cdot U_{ESG}(\\mathbf{w}) \\\\\n\\text{s.t.} \\quad & \\mathbf{w}^T \\mathbf{1} = 1 \\\\\n& \\mathbf{w} \\geq \\mathbf{0} \\\\\n& \\mathbf{C w} \\leq \\mathbf{b}\n\\end{align}\n\noù $\\mathbf{w} = (w_1, w_2, \\ldots, w_n)^T$ représente le vecteur des poids alloués aux $n$ obligations du portefeuille, $U_{fin}(\\mathbf{w})$ et $U_{ESG}(\\mathbf{w})$ sont respectivement les fonctions d'utilité financière et ESG, $\\lambda \\in [0,1]$ est le paramètre de pondération entre ces deux objectifs, et $\\mathbf{C w} \\leq \\mathbf{b}$ englobe l'ensemble des contraintes supplémentaires (sectorielles, géographiques, de duration, etc.).\n\nPour la composante financière de la fonction d'utilité, nous adoptons une formulation de type Markowitz ajustée au contexte obligataire :\n\n\\begin{equation}\nU_{fin}(\\mathbf{w}) = \\frac{\\mathbf{w}^T \\boldsymbol{\\mu} - r_f}{\\sqrt{\\mathbf{w}^T \\boldsymbol{\\Sigma} \\mathbf{w}}}\n\\end{equation}\n\noù $\\boldsymbol{\\mu}$ représente le vecteur des rendements espérés des obligations, $\\boldsymbol{\\Sigma}$ leur matrice de variance-covariance, et $r_f$ le taux sans risque.\n\nCette fonction maximise le ratio de Sharpe du portefeuille, soit le rendement excédentaire par unité de risque, une métrique particulièrement pertinente dans le contexte obligataire où les investisseurs sont typiquement averses au risque.\n\nPour la composante ESG, nous définissons :\n\n\\begin{equation}\nU_{ESG}(\\mathbf{w}) = \\frac{\\mathbf{w}^T \\mathbf{s}_{ESG} - s_{min}}{s_{max} - s_{min}}\n\\end{equation}\n\noù $\\mathbf{s}_{ESG}$ représente le vecteur des scores ESG des obligations, normalisé sur une échelle commune, et $s_{min}$ et $s_{max}$ sont respectivement les scores minimum et maximum dans l'univers d'investissement.\n\nCette formulation normalise le score ESG du portefeuille sur une échelle [0,1], facilitant son intégration avec la composante financière et permettant une interprétation intuitive de la performance ESG relative.\n\n\\subsubsection{Modélisation des contraintes spécifiques}\n\nLes contraintes incluses dans notre problème d'optimisation reflètent à la fois des considérations de gestion des risques traditionnelles et des exigences spécifiques liées à l'intégration ESG.\n\n\\paragraph{Contraintes de diversification} \nPour éviter une concentration excessive du risque, nous imposons des limites sur l'exposition maximale à chaque émetteur et secteur :\n\n\\begin{align}\nw_i &\\leq w_{max} \\quad \\forall i \\in \\{1, \\ldots, n\\} \\\\\n\\sum_{i \\in S_j} w_i &\\leq s_{max} \\quad \\forall j \\in \\{1, \\ldots, m\\}\n\\end{align}\n\noù $w_{max}$ représente la pondération maximale par émetteur (typiquement 3% dans notre implémentation), $S_j$ l'ensemble des obligations appartenant au secteur $j$, et $s_{max}$ la pondération sectorielle maximale (15% dans notre cas).\n\n\\paragraph{Contraintes de duration} \nPour contrôler le risque de taux d'intérêt du portefeuille, nous imposons des bornes sur sa duration modifiée :\n\n\\begin{equation}\nD_{min} \\leq \\sum_{i=1}^n w_i D_i \\leq D_{max}\n\\end{equation}\n\noù $D_i$ représente la duration modifiée de l'obligation $i$, et $D_{min}$ et $D_{max}$ les bornes inférieure et supérieure de duration (respectivement 3 et 8 ans dans notre implémentation).\n\n\\paragraph{Contraintes ESG minimales} \nAu-delà de l'optimisation du score ESG dans la fonction objectif, nous imposons également un score ESG minimum pour le portefeuille global :\n\n\\begin{equation}\n\\sum_{i=1}^n w_i s^{ESG}_i \\geq s^{ESG}_{min}\n\\end{equation}\n\noù $s^{ESG}_{min}$ représente le score ESG minimum requis (fixé à 55 sur une échelle de 100 dans notre implémentation).\n\nDe plus, certaines exclusions sectorielles sont formalisées comme des contraintes dures :\n\n\\begin{equation}\nw_i = 0 \\quad \\forall i \\in E\n\\end{equation}\n\noù $E$ représente l'ensemble des obligations exclues sur la base de critères ESG normatifs (controverses graves, violations du Global Compact, implication dans des armes controversées, etc.).\n\n\\paragraph{Contraintes d'exposition thématique} \nPour certains objectifs ESG spécifiques, nous introduisons des contraintes d'exposition minimale à des thématiques ciblées :\n\n\\begin{equation}\n\\sum_{i \\in T_k} w_i \\geq t_{k,min} \\quad \\forall k \\in \\{1, \\ldots, p\\}\n\\end{equation}\n\noù $T_k$ représente l'ensemble des obligations liées à la thématique $k$ (par exemple, les obligations vertes ou durables), et $t_{k,min}$ l'exposition minimale requise (10% pour les obligations vertes dans notre cas).\n\n\\subsubsection{Résolution numérique et considérations algorithmiques}\n\nLa résolution numérique du problème d'optimisation formalisé ci-dessus présente plusieurs défis techniques, notamment en raison de sa nature non linéaire et de la présence de multiples contraintes. Nous avons adopté une approche en deux étapes pour sa résolution.\n\nPremièrement, nous reformulons le problème en utilisant la technique de la frontière efficiente, en résolvant une série de problèmes paramétrés par un facteur de risque $\\delta$ :\n\n\\begin{align}\n\\max_{\\mathbf{w}} \\quad & \\mathbf{w}^T \\boldsymbol{\\mu} + (1-\\lambda) \\cdot \\delta \\cdot \\mathbf{w}^T \\mathbf{s}_{ESG} \\\\\n\\text{s.t.} \\quad & \\mathbf{w}^T \\boldsymbol{\\Sigma} \\mathbf{w} \\leq \\sigma^2_{target} \\\\\n& \\text{(autres contraintes)}\n\\end{align}\n\nCette reformulation permet d'utiliser des algorithmes d'optimisation quadratique standard tout en explorant systématiquement le front de Pareto des solutions optimales.\n\nPour la résolution numérique proprement dite, nous employons l'algorithme Sequential Least Squares Programming (SLSQP), particulièrement adapté aux problèmes non linéaires sous contraintes. Cet algorithme approche itérativement la solution en résolvant une séquence de sous-problèmes quadratiques, utilisant l'approximation de la matrice hessienne de la fonction objectif.\n\nL'implémentation algorithmique inclut également :\n\n\\begin{itemize}\n    \\item Une phase d'initialisation multi-départ pour éviter les optima locaux, en générant $k$ points de départ aléatoires respectant les contraintes et en sélectionnant la meilleure solution\n    \n    \\item Une validation croisée des résultats via la méthode du simplexe de Nelder-Mead, appliquée comme second solveur sur les meilleures solutions préliminaires\n    \n    \\item Une procédure adaptative d'ajustement du paramètre de pondération $\\lambda$ basée sur les caractéristiques de l'univers d'investissement, définie comme :\n    \n    \\begin{equation}\n    \\lambda_{adaptive} = \\lambda_{base} \\cdot \\left(1 + \\alpha \\cdot \\frac{\\sigma(\\mathbf{s}_{ESG})}{\\mu(\\mathbf{s}_{ESG})}\\right)\n    \\end{equation}\n    \n    où $\\sigma(\\mathbf{s}_{ESG})$ et $\\mu(\\mathbf{s}_{ESG})$ représentent respectivement l'écart-type et la moyenne des scores ESG dans l'univers, et $\\alpha$ un paramètre de sensibilité fixé à 0.5 dans notre implémentation.\n\\end{itemize}\n\nCette approche algorithmique nous permet d'obtenir des solutions d'optimisation robustes, évitant les problèmes de convergence locale et s'adaptant aux caractéristiques spécifiques de l'univers obligataire considéré.\n\n\\subsection{Approche retenue et implémentation}\n\nPour notre portefeuille final, nous avons adopté une approche hybride combinant plusieurs des méthodologies précédemment décrites :\n\n\\begin{itemize}\n    \\item \\textbf{Segmentation} du portefeuille en trois compartiments, correspondant aux trois types d'obligations (corporate, souveraine, verte)\n    \n    \\item \\textbf{Allocation stratégique} entre ces compartiments selon les proportions définies précédemment (60\\%, 30\\%, 10\\%)\n    \n    \\item Pour chaque compartiment :\n    \\begin{itemize}\n        \\item \\textbf{Pré-sélection} des titres selon les critères financiers et ESG définis\n        \\item \\textbf{Optimisation sous contraintes ESG}, avec une fonction objectif hybride intégrant rendement, risque et score ESG\n        \\item \\textbf{Ajustement manuel} pour garantir la représentativité sectorielle et géographique\n    \\end{itemize}\n    \n    \\item \\textbf{Rebalancement trimestriel} avec des seuils de tolérance pour limiter le turnover\n\\end{itemize}\n\nEn termes mathématiques, notre problème d'optimisation pour chaque compartiment peut être formulé comme suit :\n\n\\begin{align}\n\\max_w \\quad & (1-\\lambda) \\times \\left( \\frac{w^T \\mu - r_f}{\\sqrt{w^T \\Sigma w}} \\right) + \\lambda \\times \\left( \\frac{\\sum_{i=1}^n w_i \\times \\text{ESG}_i - \\text{ESG}_{\\text{min}}}{\\text{ESG}_{\\text{max}} - \\text{ESG}_{\\text{min}}} \\right) \\\\\n\\text{s.t.} \\quad & \\sum_{i=1}^n w_i = 1 \\\\\n& w_i \\geq 0, \\quad \\forall i \\in \\{1,...,n\\} \\\\\n& w_i \\leq w_{\\text{max}}, \\quad \\forall i \\in \\{1,...,n\\} \\\\\n& \\sum_{i \\in S_j} w_i \\leq s_{\\text{max}}, \\quad \\forall j \\in \\{1,...,m\\} \\\\\n& L_{\\text{dur}} \\leq \\sum_{i=1}^n w_i \\times D_i \\leq U_{\\text{dur}} \\\\\n& \\sum_{i=1}^n w_i \\times \\text{ESG}_i \\geq \\text{ESG}_{\\text{min}}\n\\end{align}\n\noù :\n\\begin{itemize}\n    \\item $\\lambda = 0.3$ est le paramètre de pondération entre objectifs financiers et ESG\n    \\item $w_{\\text{max}} = 3\\%$ est la pondération maximale par émetteur\n    \\item $S_j$ représente l'ensemble des titres appartenant au secteur $j$\n    \\item $s_{\\text{max}} = 15\\%$ est la pondération sectorielle maximale\n    \\item $L_{\\text{dur}} = 3$ et $U_{\\text{dur}} = 8$ sont les bornes inférieure et supérieure de duration\n    \\item $\\text{ESG}_{\\text{min}} = 55$ est le score ESG minimal du portefeuille (sur une échelle de 0 à 100)\n\\end{itemize}\n\nLa résolution de ce problème d'optimisation a été réalisée à l'aide de l'algorithme Sequential Least Squares Programming (SLSQP), particulièrement adapté aux problèmes non linéaires sous contraintes, avec une validation croisée des résultats via la méthode du simplexe de Nelder-Mead pour éviter les optima locaux.\n\n\\subsection{Implémentation algorithmique et complexité computationnelle}\n\nL'implémentation pratique de notre approche d'optimisation a nécessité plusieurs adaptations algorithmiques pour gérer efficacement la complexité computationnelle inhérente à ce type de problème multi-objectif sous contraintes multiples, particulièrement dans le contexte d'un univers obligataire de grande dimension.\n\n\\subsubsection{Décomposition hiérarchique du problème}\n\nFace à la difficulté de résoudre directement le problème global avec plusieurs centaines d'obligations candidates, nous avons adopté une approche de décomposition hiérarchique structurée comme suit :\n\n\\begin{algorithm}\n\\caption{Optimisation hiérarchique du portefeuille obligataire ESG}\n\\begin{algorithmic}[1]\n\\REQUIRE Univers d'obligations $\\Omega$, Matrices de covariance $\\Sigma$, Rendements attendus $\\mu$, Scores ESG $s_{ESG}$\n\\ENSURE Allocation optimale $w^*$\n\n\\STATE Segmenter $\\Omega$ en compartiments $\\Omega_1, \\Omega_2, \\ldots, \\Omega_K$ selon les types d'obligations\n\\STATE Déterminer l'allocation stratégique $\\alpha = (\\alpha_1, \\alpha_2, \\ldots, \\alpha_K)$ entre compartiments\n\n\\FOR{chaque compartiment $k \\in \\{1, 2, \\ldots, K\\}$}\n    \\STATE Appliquer les filtres d'exclusion ESG: $\\Omega_k' = \\Omega_k \\setminus E_k$\n    \\STATE Appliquer les critères de pré-sélection financière et ESG pour obtenir $\\Omega_k''$\n    \\STATE Résoudre le problème d'optimisation sur $\\Omega_k''$ pour obtenir $w_k^*$\n\\ENDFOR\n\n\\STATE Construire l'allocation finale: $w^* = (\\alpha_1 w_1^*, \\alpha_2 w_2^*, \\ldots, \\alpha_K w_K^*)$\n\\STATE Appliquer les ajustements manuels pour garantir la représentativité et obtenir $\\tilde{w}^*$\n\\RETURN $\\tilde{w}^*$\n\\end{algorithmic}\n\\end{algorithm}\n\nCette décomposition réduit significativement la dimension de chaque sous-problème d'optimisation, permettant une résolution plus efficace tout en préservant la structure globale de l'allocation.\n\n\\subsubsection{Estimation robuste des matrices de covariance}\n\nL'optimisation de type Markowitz est notoirement sensible à l'estimation de la matrice de covariance. Pour améliorer la robustesse de notre implémentation, nous avons adopté l'estimateur de rétrécissement (shrinkage estimator) de Ledoit-Wolf :\n\n\\begin{equation}\n\\hat{\\Sigma}_{LW} = (1 - \\delta) \\hat{\\Sigma}_{sample} + \\delta \\hat{\\Sigma}_{target}\n\\end{equation}\n\noù $\\hat{\\Sigma}_{sample}$ est l'estimateur classique de la matrice de covariance échantillonnale, $\\hat{\\Sigma}_{target}$ une matrice cible (typiquement diagonale ou à corrélation constante), et $\\delta \\in [0,1]$ le paramètre de rétrécissement optimal déterminé analytiquement pour minimiser l'erreur quadratique moyenne.\n\nPour les rendements obligataires, nous avons utilisé une matrice cible structurée par les facteurs de risque principaux :\n\n\\begin{equation}\n\\hat{\\Sigma}_{target} = BB^T + D\n\\end{equation}\n\noù $B$ est la matrice des charges factorielles estimée par analyse en composantes principales sur les rendements historiques (conservant les 5 premiers facteurs), et $D$ une matrice diagonale des variances résiduelles.\n\n\\subsubsection{Optimisation numérique et complexité}  \n\nLa résolution numérique de chaque sous-problème d'optimisation utilise l'algorithme SLSQP avec plusieurs améliorations techniques :\n\n\\begin{itemize}\n    \\item Reformulation du problème pour améliorer son conditionnement numérique, notamment par la normalisation des variables et l'élimination des contraintes redondantes\n    \n    \\item Implémentation d'une recherche multi-départ avec 50 points initiaux générés par échantillonnage de quasi-Monte Carlo (séquence de Sobol) pour une exploration plus systématique de l'espace des solutions\n    \n    \\item Parallélisation des calculs sur 8 cœurs pour les différents points de départ, réduisant significativement le temps de calcul global\n\\end{itemize}\n\nLa complexité computationnelle de notre approche peut être analysée comme suit :\n\n\\begin{itemize}\n    \\item Complexité temporelle : $O(K \\cdot M \\cdot I \\cdot N^3)$ où $K$ est le nombre de compartiments, $M$ le nombre de points de départ, $I$ le nombre typique d'itérations de l'algorithme d'optimisation, et $N$ la taille moyenne de chaque sous-problème\n    \n    \\item Complexité spatiale : $O(N^2)$ principalement dominée par le stockage des matrices de covariance\n\\end{itemize}\n\nPour notre implémentation avec $K=3$ compartiments, des sous-problèmes de taille moyenne $N \\approx 100$ obligations, et $M=50$ points de départ, le temps de calcul total était d'environ 15 minutes sur un serveur standard avec 32 Go de RAM et un processeur Intel Xeon à 3.5 GHz.\n\n\\subsubsection{Rebalancement et contrôle du turnover}\n\nNotre stratégie de rebalancement trimestriel intègre une composante de contrôle du turnover pour limiter les coûts de transaction tout en maintenant l'alignement du portefeuille avec ses objectifs financiers et ESG. Formellement, nous introduisons une pénalité de turnover dans la fonction objectif lors des rebalancements :\n\n\\begin{equation}\nU_{rebal}(\\mathbf{w}) = U(\\mathbf{w}) - \\kappa \\cdot \\sum_{i=1}^n |w_i - w_i^{prev}|\n\\end{equation}\n\noù $U(\\mathbf{w})$ est la fonction d'utilité standard définie précédemment, $w_i^{prev}$ le poids de l'obligation $i$ avant rebalancement, et $\\kappa$ un paramètre de pénalité fixé à 0.1 dans notre implémentation.\n\nDe plus, nous implémentons des seuils de tolérance pour éviter les transactions de faible amplitude :\n\n\\begin{equation}\nw_i^{new} = \\begin{cases}\nw_i^{opt} & \\text{si } |w_i^{opt} - w_i^{prev}| > \\epsilon \\\\\nw_i^{prev} & \\text{sinon}\n\\end{cases}\n\\end{equation}\n\noù $w_i^{opt}$ est le poids optimal issu de l'optimisation, $w_i^{new}$ le poids effectivement implémenté, et $\\epsilon$ un seuil de tolérance fixé à 0.5% dans notre implémentation.\n\nCette approche algorithmique complète nous a permis de construire un portefeuille obligataire optimisé intégrant efficacement les considérations ESG tout en maintenant des caractéristiques financières attrayantes, comme détaillé dans les sections suivantes.\n\n\\section{Présentation des caractéristiques du portefeuille étudié}