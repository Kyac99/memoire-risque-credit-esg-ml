\chapter{Conception et analyse du portefeuille obligataire ESG}

\section{Introduction et contexte du portefeuille obligataire}

La construction d'un portefeuille obligataire intégrant les critères environnementaux, sociaux et de gouvernance (ESG) représente un défi méthodologique majeur dans la finance moderne. Contrairement aux actions, les obligations présentent des caractéristiques spécifiques qui influencent fondamentalement l'approche d'optimisation : flux de revenus prédéterminés, sensibilité aux variations de taux d'intérêt, risque de crédit variable selon l'émetteur, et maturité définie. Ces spécificités nécessitent une adaptation des méthodologies d'optimisation traditionnelles pour incorporer efficacement les considérations ESG sans compromettre les objectifs financiers.

Le présent chapitre présente notre approche de construction d'un portefeuille obligataire optimisé sous contraintes ESG, développée dans le cadre de cette recherche. Notre méthodologie s'appuie sur une formulation mathématique rigoureuse du problème d'optimisation multi-objectif, intégrant simultanément performance financière et performance extra-financière. L'originalité de notre approche réside dans la décomposition hiérarchique du problème d'optimisation et l'utilisation d'algorithmes numériques avancés adaptés aux spécificités du marché obligataire.

\section{Présentation de l'univers d'investissement et des données}

\subsection{Constitution de l'univers obligataire}

Notre univers d'investissement est constitué de 847 obligations sélectionnées selon des critères stricts de liquidité, de qualité de crédit et de disponibilité des données ESG. La composition de cet univers reflète la diversité du marché obligataire mondial tout en respectant les contraintes opérationnelles d'un gestionnaire de portefeuille institutionnel.

\subsubsection{Critères de sélection des titres}

Les obligations incluses dans notre univers répondent aux critères suivants :

\begin{itemize}
    \item \textbf{Liquidité minimale} : volume de transactions quotidien moyen supérieur à 10 millions d'euros sur les 12 derniers mois
    \item \textbf{Maturité résiduelle} : comprise entre 1 et 30 ans pour assurer une diversification temporelle appropriée
    \item \textbf{Notation minimale} : rating BBB- ou équivalent pour limiter le risque de crédit
    \item \textbf{Devise} : obligations libellées en euros pour éliminer le risque de change
    \item \textbf{Disponibilité ESG} : existence d'un score ESG complet et récent (moins de 12 mois) fourni par MSCI ESG Research
\end{itemize}

\subsubsection{Segmentation par type d'obligations}

L'univers est segmenté en trois catégories principales selon la nature de l'émetteur et les caractéristiques spécifiques des instruments :

\paragraph{Obligations corporates (508 titres - 60\% de l'allocation)}
Les obligations d'entreprises constituent le segment le plus important de notre univers. Cette catégorie inclut :
\begin{itemize}
    \item \textbf{Secteurs représentés} : 24 secteurs selon la classification GICS niveau 2, avec une représentation équilibrée des services financiers (18\%), de l'industrie (15\%), de la consommation discrétionnaire (12\%), et de la technologie (10\%)
    \item \textbf{Qualité de crédit} : distribution centrée sur les ratings A- à BBB+, reflétant un profil risque-rendement équilibré
    \item \textbf{Duration moyenne} : 5.2 ans avec une dispersion de 2.1 à 12.8 ans
    \item \textbf{Rendement moyen} : 3.4\% avec un spread crédit moyen de 185 points de base par rapport aux obligations souveraines équivalentes
\end{itemize}

\paragraph{Obligations souveraines (254 titres - 30\% de l'allocation)}
Le segment souverain comprend exclusivement des obligations émises par les États membres de la zone euro et quelques émetteurs quasi-souverains de qualité équivalente :
\begin{itemize}
    \item \textbf{Émetteurs principaux} : Allemagne (28\%), France (22\%), Italie (18\%), Espagne (15\%), Pays-Bas (8\%), autres (9\%)
    \item \textbf{Profil de risque} : notation AAA à BBB-, avec une concentration sur les émetteurs AA/A
    \item \textbf{Duration moyenne} : 7.1 ans, généralement supérieure aux corporates
    \item \textbf{Rendement moyen} : 2.1\%, servant de référence "sans risque" pour l'évaluation du spread des autres segments
\end{itemize}

\paragraph{Obligations vertes et durables (85 titres - 10\% de l'allocation)}
Ce segment, en forte croissance, regroupe les obligations dont les fonds sont spécifiquement dédiés à des projets environnementaux ou sociaux :
\begin{itemize}
    \item \textbf{Green bonds} : 65 titres finançant des projets de transition énergétique, d'efficacité énergétique, de transport propre et de gestion de l'eau
    \item \textbf{Social bonds} : 12 titres soutenant l'accès à l'éducation, aux soins de santé et au logement abordable
    \item \textbf{Sustainability bonds} : 8 titres combinant objectifs environnementaux et sociaux
    \item \textbf{Premium ESG} : spread moyen de -15 points de base par rapport aux obligations conventionnelles comparables, reflétant une demande institutionnelle soutenue
\end{itemize}

\subsection{Sources et traitement des données}

\subsubsection{Données financières}

Les données de marché proviennent de Bloomberg Terminal et sont mises à jour quotidiennement. Pour chaque obligation, nous collectons :

\begin{itemize}
    \item \textbf{Prix et rendements} : cours de clôture, rendement à l'échéance, spread par rapport à la courbe de référence
    \item \textbf{Caractéristiques techniques} : duration modifiée, convexité, sensibilité aux spreads de crédit
    \item \textbf{Données d'émission} : nominal émis, date d'émission et d'échéance, taux de coupon, fréquence de versement
    \item \textbf{Notations} : ratings des trois agences principales (Moody's, S\&P, Fitch) avec réconciliation selon une échelle numérique commune
\end{itemize}

Un processus de nettoyage systématique élimine les observations aberrantes et traite les données manquantes par interpolation ou extrapolation selon des règles préétablies.

\subsubsection{Données ESG}

Les scores ESG proviennent de MSCI ESG Research et sont complétés par des données Sustainalytics pour validation croisée. Notre méthodologie d'agrégation combine :

\begin{itemize}
    \item \textbf{Score E (Environnement)} : intensité carbone, gestion des déchets, efficacité énergétique, impact sur la biodiversité
    \item \textbf{Score S (Social)} : relations sociales, sécurité des produits, gestion du capital humain, impact communautaire
    \item \textbf{Score G (Gouvernance)} : structure du conseil d'administration, transparence financière, éthique des affaires, droits des actionnaires
    \item \textbf{Score agrégé} : moyenne pondérée des trois composantes avec des poids de 40\%, 30\% et 30\% respectivement, reflétant l'importance croissante des enjeux climatiques
\end{itemize}

La normalisation des scores sur une échelle 0-100 facilite leur intégration dans les fonctions d'optimisation et permet une interprétation intuitive des résultats.

\subsubsection{Données macroéconomiques complémentaires}

Pour enrichir notre analyse, nous intégrons également :

\begin{itemize}
    \item \textbf{Courbes de taux} : taux swap euro de 1 mois à 30 ans, mis à jour quotidiennement
    \item \textbf{Indicateurs de volatilité} : indice MOVE (équivalent obligataire du VIX), spreads bid-ask moyens par segment
    \item \textbf{Flux de capitaux} : données d'émission primaire et de rachats secondaires pour anticiper les évolutions d'offre
    \item \textbf{Facteurs ESG macroéconomiques} : réglementations environnementales, taxonomie européenne, stress tests climatiques des banques centrales
\end{itemize}

\section{Méthodologie quantitative d'optimisation de portefeuille avec contraintes ESG}

\subsection{Cadre théorique d'optimisation sous contraintes ESG}

L'intégration des critères ESG dans la construction d'un portefeuille obligataire nécessite une adaptation des cadres d'optimisation traditionnels pour incorporer explicitement ces considérations extra-financières. Cette section présente le cadre théorique et mathématique sous-tendant notre approche d'optimisation.

\subsubsection{Formulation du problème d'optimisation multi-objectif}

La construction d'un portefeuille obligataire intégrant les critères ESG peut être formulée comme un problème d'optimisation multi-objectif, cherchant à équilibrer performance financière et performance extra-financière. Formellement, nous cherchons à résoudre :

\begin{align}
\max_{\mathbf{w}} \quad & U(\mathbf{w}) = (1-\lambda) \cdot U_{fin}(\mathbf{w}) + \lambda \cdot U_{ESG}(\mathbf{w}) \\
\text{s.t.} \quad & \mathbf{w}^T \mathbf{1} = 1 \\
& \mathbf{w} \geq \mathbf{0} \\
& \mathbf{C w} \leq \mathbf{b}
\end{align}

où $\mathbf{w} = (w_1, w_2, \ldots, w_n)^T$ représente le vecteur des poids alloués aux $n$ obligations du portefeuille, $U_{fin}(\mathbf{w})$ et $U_{ESG}(\mathbf{w})$ sont respectivement les fonctions d'utilité financière et ESG, $\lambda \in [0,1]$ est le paramètre de pondération entre ces deux objectifs, et $\mathbf{C w} \leq \mathbf{b}$ englobe l'ensemble des contraintes supplémentaires (sectorielles, géographiques, de duration, etc.).

Pour la composante financière de la fonction d'utilité, nous adoptons une formulation de type Markowitz ajustée au contexte obligataire :

\begin{equation}
U_{fin}(\mathbf{w}) = \frac{\mathbf{w}^T \boldsymbol{\mu} - r_f}{\sqrt{\mathbf{w}^T \boldsymbol{\Sigma} \mathbf{w}}}
\end{equation}

où $\boldsymbol{\mu}$ représente le vecteur des rendements espérés des obligations, $\boldsymbol{\Sigma}$ leur matrice de variance-covariance, et $r_f$ le taux sans risque.

Cette fonction maximise le ratio de Sharpe du portefeuille, soit le rendement excédentaire par unité de risque, une métrique particulièrement pertinente dans le contexte obligataire où les investisseurs sont typiquement averses au risque.

Pour la composante ESG, nous définissons :

\begin{equation}
U_{ESG}(\mathbf{w}) = \frac{\mathbf{w}^T \mathbf{s}_{ESG} - s_{min}}{s_{max} - s_{min}}
\end{equation}

où $\mathbf{s}_{ESG}$ représente le vecteur des scores ESG des obligations, normalisé sur une échelle commune, et $s_{min}$ et $s_{max}$ sont respectivement les scores minimum et maximum dans l'univers d'investissement.

Cette formulation normalise le score ESG du portefeuille sur une échelle [0,1], facilitant son intégration avec la composante financière et permettant une interprétation intuitive de la performance ESG relative.

\paragraph{Extension avec facteurs de risque multiples}
La formulation précédente peut être enrichie pour tenir compte de multiples facteurs de risque, particulièrement pertinents dans le contexte obligataire. Le modèle d'utilité financière s'étend alors à :

\begin{equation}
U_{fin}(\mathbf{w}) = \frac{\mathbf{w}^T \boldsymbol{\mu} - r_f}{\sqrt{\mathbf{w}^T \left(\sum_{k=1}^{K} \beta_k \beta_k^T \sigma_k^2 + \mathbf{D}\right) \mathbf{w}}}
\end{equation}

où :
\begin{itemize}
    \item $\beta_k$ représente le vecteur des expositions des obligations au facteur de risque $k$
    \item $\sigma_k^2$ est la variance du facteur $k$
    \item $\mathbf{D}$ est une matrice diagonale des variances idiosyncratiques
\end{itemize}

Ces facteurs peuvent inclure :
\begin{itemize}
    \item Facteur de duration (sensibilité aux variations des taux d'intérêt)
    \item Facteur de crédit (sensibilité aux variations des spreads)
    \item Facteur de convexité (sensibilité aux mouvements non-linéaires de la courbe)
    \item Facteurs sectoriels et géographiques
\end{itemize}

Cette décomposition factorielle permet une attribution plus précise et une gestion plus granulaire du risque, ainsi qu'une estimation plus robuste de la matrice de variance-covariance, particulièrement pour les univers obligataires de grande dimension.

\paragraph{Modélisation avancée de l'utilité ESG}
La fonction d'utilité ESG peut être raffinée pour intégrer différentes dimensions et préférences spécifiques :

\begin{equation}
U_{ESG}(\mathbf{w}) = \sum_{j=1}^{3} \omega_j \frac{\mathbf{w}^T \mathbf{s}_{j} - s_{j,min}}{s_{j,max} - s_{j,min}} - \kappa \frac{\sum_{i=1}^{n} w_i \cdot \mathbb{I}(\mathbf{s}_{i} < s_{thresh})}{\sum_{i=1}^{n} w_i}
\end{equation}

où :
\begin{itemize}
    \item $\mathbf{s}_{j}$ représente le vecteur des scores pour la dimension ESG $j$ (environnement, social, gouvernance)
    \item $\omega_j$ est le poids attribué à la dimension $j$, avec $\sum_{j=1}^{3} \omega_j = 1$
    \item Le second terme pénalise la proportion du portefeuille investie dans des obligations dont le score ESG est inférieur à un seuil $s_{thresh}$, avec $\mathbb{I}(\cdot)$ la fonction indicatrice
    \item $\kappa$ est un paramètre de pénalité ajustable
\end{itemize}

Cette formulation permet de mieux refléter les préférences spécifiques des investisseurs, comme une importance relative différente accordée aux trois piliers ESG ou une aversion particulière aux émetteurs de faible qualité ESG.

\subsubsection{Modélisation des contraintes spécifiques}

Les contraintes incluses dans notre problème d'optimisation reflètent à la fois des considérations de gestion des risques traditionnelles et des exigences spécifiques liées à l'intégration ESG.

\paragraph{Contraintes de diversification} 
Pour éviter une concentration excessive du risque, nous imposons des limites sur l'exposition maximale à chaque émetteur et secteur :

\begin{align}
w_i &\leq w_{max} \quad \forall i \in \{1, \ldots, n\} \\
\sum_{i \in S_j} w_i &\leq s_{max} \quad \forall j \in \{1, \ldots, m\}
\end{align}

où $w_{max}$ représente la pondération maximale par émetteur (typiquement 3\% dans notre implémentation), $S_j$ l'ensemble des obligations appartenant au secteur $j$, et $s_{max}$ la pondération sectorielle maximale (15\% dans notre cas).

\paragraph{Extension de diversification avec contraintes de concentration}
Au-delà des limites simples par émetteur et secteur, nous pouvons introduire des contraintes plus sophistiquées contrôlant la concentration globale du portefeuille :

\begin{equation}
\sum_{i=1}^{n} w_i^2 \leq HHI_{max}
\end{equation}

où $HHI_{max}$ représente le niveau maximal acceptable de l'indice de Herfindahl-Hirschman, mesurant la concentration du portefeuille.

De plus, nous pouvons contrôler la diversification effective via une contrainte sur le nombre effectif d'émetteurs indépendants ($NEE$) :

\begin{equation}
NEE = \frac{1}{\sum_{i=1}^{n} \sum_{j=1}^{n} w_i w_j \rho_{ij}} \geq NEE_{min}
\end{equation}

où $\rho_{ij}$ est la corrélation entre les rendements des émetteurs $i$ et $j$, et $NEE_{min}$ est le nombre minimal d'émetteurs effectifs requis (par exemple, 10).

Cette contrainte, plus sophistiquée que les limites nominales, assure une véritable diversification tenant compte des corrélations entre émetteurs, particulièrement importante dans le contexte obligataire où certains émetteurs peuvent être fortement corrélés malgré leur appartenance à des secteurs différents.

\paragraph{Contraintes de duration} 
Pour contrôler le risque de taux d'intérêt du portefeuille, nous imposons des bornes sur sa duration modifiée :

\begin{equation}
D_{min} \leq \sum_{i=1}^n w_i D_i \leq D_{max}
\end{equation}

où $D_i$ représente la duration modifiée de l'obligation $i$, et $D_{min}$ et $D_{max}$ les bornes inférieure et supérieure de duration (respectivement 3 et 8 ans dans notre implémentation).

\paragraph{Extension du contrôle du risque de taux}
Le contrôle du risque de taux peut être étendu pour capturer non seulement la sensibilité globale aux variations parallèles des taux (duration), mais aussi les expositions aux changements de forme de la courbe des taux :

\begin{align}
\left|\sum_{i=1}^{n} w_i \cdot K_{1,i}\right| &\leq K_{1,max} \quad \text{(niveau)} \\
\left|\sum_{i=1}^{n} w_i \cdot K_{2,i}\right| &\leq K_{2,max} \quad \text{(pente)} \\
\left|\sum_{i=1}^{n} w_i \cdot K_{3,i}\right| &\leq K_{3,max} \quad \text{(courbure)}
\end{align}

où $K_{j,i}$ représente la sensibilité de l'obligation $i$ au $j$-ème facteur principal de la courbe des taux (typiquement dérivé d'une analyse en composantes principales des mouvements historiques), et $K_{j,max}$ la limite d'exposition à ce facteur.

Cette approche multifactorielle du risque de taux permet une gestion plus granulaire et une protection contre des scénarios de déformation de la courbe qui affectent différemment les obligations selon leur positionnement sur l'échelle des maturités.

\paragraph{Contraintes ESG minimales} 
Au-delà de l'optimisation du score ESG dans la fonction objectif, nous imposons également un score ESG minimum pour le portefeuille global :

\begin{equation}
\sum_{i=1}^n w_i s^{ESG}_i \geq s^{ESG}_{min}
\end{equation}

où $s^{ESG}_{min}$ représente le score ESG minimum requis (fixé à 55 sur une échelle de 100 dans notre implémentation).

De plus, certaines exclusions sectorielles sont formalisées comme des contraintes dures :

\begin{equation}
w_i = 0 \quad \forall i \in E
\end{equation}

où $E$ représente l'ensemble des obligations exclues sur la base de critères ESG normatifs (controverses graves, violations du Global Compact, implication dans des armes controversées, etc.).

\paragraph{Contraintes d'impact carbone et de transition climatique}
Les considérations climatiques peuvent être formalisées plus précisément via des contraintes sur l'empreinte carbone et l'alignement avec une trajectoire de décarbonation :

\begin{align}
\frac{\sum_{i=1}^{n} w_i \cdot IC_i}{\sum_{i=1}^{n} w_i \cdot CA_i} &\leq IC_{max} \quad \text{(intensité carbone)} \\
\sum_{i=1}^{n} w_i \cdot AT_i &\geq AT_{min} \quad \text{(alignement transition)}
\end{align}

où :
\begin{itemize}
    \item $IC_i$ représente les émissions de gaz à effet de serre (scopes 1 et 2) de l'émetteur $i$
    \item $CA_i$ est le chiffre d'affaires de l'émetteur $i$
    \item $IC_{max}$ est l'intensité carbone maximale tolérée pour le portefeuille
    \item $AT_i$ est un score d'alignement de l'émetteur $i$ avec une trajectoire de décarbonation compatible avec l'Accord de Paris
    \item $AT_{min}$ est le score minimum d'alignement requis pour le portefeuille
\end{itemize}

Ces contraintes permettent de traduire concrètement des engagements de décarbonation et d'alignement climatique dans le processus d'optimisation du portefeuille.

\paragraph{Contraintes d'exposition thématique} 
Pour certains objectifs ESG spécifiques, nous introduisons des contraintes d'exposition minimale à des thématiques ciblées :

\begin{equation}
\sum_{i \in T_k} w_i \geq t_{k,min} \quad \forall k \in \{1, \ldots, p\}
\end{equation}

où $T_k$ représente l'ensemble des obligations liées à la thématique $k$ (par exemple, les obligations vertes ou durables), et $t_{k,min}$ l'exposition minimale requise (10\% pour les obligations vertes dans notre cas).

\paragraph{Extension des expositions thématiques avec objectifs d'impact}
Les contraintes thématiques peuvent être étendues pour cibler non seulement des classes d'actifs spécifiques mais aussi des objectifs d'impact mesurables :

\begin{equation}
\sum_{i=1}^{n} w_i \cdot I_{j,i} \geq I_{j,min} \quad \forall j \in \{1, \ldots, q\}
\end{equation}

où $I_{j,i}$ représente la contribution de l'émetteur $i$ à l'indicateur d'impact $j$ (par exemple, emplois créés, réduction d'émissions, économie d'eau, accès à l'éducation, etc.), et $I_{j,min}$ le niveau minimal requis pour cet indicateur.

Cette approche permet d'aligner le portefeuille avec des objectifs d'impact spécifiques et mesurables, au-delà des simples scores ESG agrégés, et de contribuer directement à des Objectifs de Développement Durable particuliers.

\subsubsection{Résolution numérique et considérations algorithmiques}

La résolution numérique du problème d'optimisation formalisé ci-dessus présente plusieurs défis techniques, notamment en raison de sa nature non linéaire et de la présence de multiples contraintes. Nous avons adopté une approche en deux étapes pour sa résolution.

Premièrement, nous reformulons le problème en utilisant la technique de la frontière efficiente, en résolvant une série de problèmes paramétrés par un facteur de risque $\delta$ :

\begin{align}
\max_{\mathbf{w}} \quad & \mathbf{w}^T \boldsymbol{\mu} + (1-\lambda) \cdot \delta \cdot \mathbf{w}^T \mathbf{s}_{ESG} \\
\text{s.t.} \quad & \mathbf{w}^T \boldsymbol{\Sigma} \mathbf{w} \leq \sigma^2_{target} \\
& \text{(autres contraintes)}
\end{align}

Cette reformulation permet d'utiliser des algorithmes d'optimisation quadratique standard tout en explorant systématiquement le front de Pareto des solutions optimales.

\paragraph{Décomposition par dualité de Lagrange}
Pour traiter efficacement les nombreuses contraintes, nous pouvons exploiter la décomposition par dualité de Lagrange. Le Lagrangien associé à notre problème s'écrit :

\begin{align}
L(\mathbf{w}, \boldsymbol{\lambda}, \boldsymbol{\mu}, \nu) = &-\mathbf{w}^T \boldsymbol{\mu} - (1-\lambda) \cdot \delta \cdot \mathbf{w}^T \mathbf{s}_{ESG} + \frac{\gamma}{2}\mathbf{w}^T \boldsymbol{\Sigma} \mathbf{w} \\
&+ \boldsymbol{\lambda}^T (\mathbf{C}\mathbf{w} - \mathbf{b}) - \boldsymbol{\mu}^T \mathbf{w} + \nu(\mathbf{w}^T\mathbf{1} - 1)
\end{align}

où $\boldsymbol{\lambda} \geq \mathbf{0}$, $\boldsymbol{\mu} \geq \mathbf{0}$ et $\nu$ sont les multiplicateurs de Lagrange associés respectivement aux contraintes d'inégalité $\mathbf{C}\mathbf{w} \leq \mathbf{b}$, de non-négativité $\mathbf{w} \geq \mathbf{0}$ et de somme unitaire $\mathbf{w}^T\mathbf{1} = 1$.

La résolution du problème dual :

\begin{equation}
\max_{\boldsymbol{\lambda} \geq \mathbf{0}, \boldsymbol{\mu} \geq \mathbf{0}, \nu} \min_{\mathbf{w}} L(\mathbf{w}, \boldsymbol{\lambda}, \boldsymbol{\mu}, \nu)
\end{equation}

permet de décomposer le problème original en sous-problèmes plus facilement traitables, en particulier pour les portefeuilles de grande dimension.

Pour la résolution numérique proprement dite, nous employons l'algorithme Sequential Least Squares Programming (SLSQP), particulièrement adapté aux problèmes non linéaires sous contraintes. Cet algorithme approche itérativement la solution en résolvant une séquence de sous-problèmes quadratiques, utilisant l'approximation de la matrice hessienne de la fonction objectif.

\subsection{Approche retenue et implémentation}

Pour notre portefeuille final, nous avons adopté une approche hybride combinant plusieurs des méthodologies précédemment décrites :

\begin{itemize}
    \item \textbf{Segmentation} du portefeuille en trois compartiments, correspondant aux trois types d'obligations (corporate, souveraine, verte)
    
    \item \textbf{Allocation stratégique} entre ces compartiments selon les proportions définies précédemment (60\%, 30\%, 10\%)
    
    \item Pour chaque compartiment :
    \begin{itemize}
        \item \textbf{Pré-sélection} des titres selon les critères financiers et ESG définis
        \item \textbf{Optimisation sous contraintes ESG}, avec une fonction objectif hybride intégrant rendement, risque et score ESG
        \item \textbf{Ajustement manuel} pour garantir la représentativité sectorielle et géographique
    \end{itemize}
    
    \item \textbf{Rebalancement trimestriel} avec des seuils de tolérance pour limiter le turnover
\end{itemize}

En termes mathématiques, notre problème d'optimisation pour chaque compartiment peut être formulé comme suit :

\begin{align}
\max_w \quad & (1-\lambda) \times \left( \frac{w^T \mu - r_f}{\sqrt{w^T \Sigma w}} \right) + \lambda \times \left( \frac{\sum_{i=1}^n w_i \times \text{ESG}_i - \text{ESG}_{\text{min}}}{\text{ESG}_{\text{max}} - \text{ESG}_{\text{min}}} \right) \\
\text{s.t.} \quad & \sum_{i=1}^n w_i = 1 \\
& w_i \geq 0, \quad \forall i \in \{1,...,n\} \\
& w_i \leq w_{\text{max}}, \quad \forall i \in \{1,...,n\} \\
& \sum_{i \in S_j} w_i \leq s_{\text{max}}, \quad \forall j \in \{1,...,m\} \\
& L_{\text{dur}} \leq \sum_{i=1}^n w_i \times D_i \leq U_{\text{dur}} \\
& \sum_{i=1}^n w_i \times \text{ESG}_i \geq \text{ESG}_{\text{min}}
\end{align}

où :
\begin{itemize}
    \item $\lambda = 0.3$ est le paramètre de pondération entre objectifs financiers et ESG
    \item $w_{\text{max}} = 3\%$ est la pondération maximale par émetteur
    \item $S_j$ représente l'ensemble des titres appartenant au secteur $j$
    \item $s_{\text{max}} = 15\%$ est la pondération sectorielle maximale
    \item $L_{\text{dur}} = 3$ et $U_{\text{dur}} = 8$ sont les bornes inférieure et supérieure de duration
    \item $\text{ESG}_{\text{min}} = 55$ est le score ESG minimal du portefeuille (sur une échelle de 0 à 100)
\end{itemize}

\section{Présentation et analyse du portefeuille optimal}

\subsection{Composition et caractéristiques générales}

Le portefeuille optimal obtenu à partir de notre méthodologie d'optimisation présente les caractéristiques suivantes, reflétant l'équilibre recherché entre performance financière et critères ESG.

\subsubsection{Allocation par type d'obligations}

La répartition finale du portefeuille respecte globalement l'allocation stratégique prédéfinie tout en intégrant les ajustements résultant de l'optimisation :

\begin{table}[h!]
\centering
\begin{tabular}{lccc}
\hline
\textbf{Type d'obligation} & \textbf{Allocation cible} & \textbf{Allocation réalisée} & \textbf{Nombre de titres} \\
\hline
Obligations corporates & 60.0\% & 58.7\% & 127 \\
Obligations souveraines & 30.0\% & 31.8\% & 73 \\
Obligations vertes/durables & 10.0\% & 9.5\% & 24 \\
\hline
\textbf{Total} & \textbf{100.0\%} & \textbf{100.0\%} & \textbf{224} \\
\hline
\end{tabular}
\caption{Allocation du portefeuille par type d'obligations}
\end{table}

Les légers écarts par rapport aux allocations cibles s'expliquent par l'optimisation sous contraintes, qui privilégie les titres présentant le meilleur ratio rendement-risque ajusté ESG dans chaque segment.

\subsubsection{Diversification sectorielle}

La diversification sectorielle du portefeuille corporate reflète une approche équilibrée, évitant les concentrations excessives tout en respectant les contraintes d'optimisation :

\begin{table}[h!]
\centering
\begin{tabular}{lcc}
\hline
\textbf{Secteur} & \textbf{Allocation (\%)} & \textbf{Score ESG moyen} \\
\hline
Services financiers & 12.3 & 67.8 \\
Industrie & 10.8 & 72.1 \\
Consommation discrétionnaire & 8.9 & 69.4 \\
Technologie & 7.2 & 74.6 \\
Santé & 6.1 & 71.2 \\
Consommation de base & 5.4 & 68.9 \\
Énergie & 4.8 & 58.3 \\
Utilities & 4.2 & 76.2 \\
Autres secteurs & 3.0 & 70.5 \\
\hline
\textbf{Total corporate} & \textbf{62.7} & \textbf{70.1} \\
\hline
\end{tabular}
\caption{Répartition sectorielle du segment corporate}
\end{table}

On observe une sous-pondération du secteur énergétique (4.8\% vs 8.1\% dans l'indice de référence), compensée par une surpondération des secteurs technologie et utilities, qui présentent des scores ESG supérieurs à la moyenne.

\subsubsection{Diversification géographique des émetteurs souverains}

La répartition géographique du segment souverain privilégie les émetteurs présentant le meilleur équilibre crédit-ESG :

\begin{table}[h!]
\centering
\begin{tabular}{lccc}
\hline
\textbf{Pays} & \textbf{Allocation (\%)} & \textbf{Rating moyen} & \textbf{Score ESG} \\
\hline
Allemagne & 8.9 & AAA & 82.4 \\
France & 7.2 & AA & 79.1 \\
Pays-Bas & 4.1 & AAA & 85.2 \\
Autriche & 3.2 & AA+ & 81.7 \\
Belgique & 2.8 & AA- & 78.8 \\
Finlande & 2.1 & AA+ & 87.3 \\
Espagne & 1.9 & A+ & 74.2 \\
Italie & 1.6 & BBB & 69.5 \\
\hline
\textbf{Total souverain} & \textbf{31.8} & \textbf{AA-} & \textbf{79.8} \\
\hline
\end{tabular}
\caption{Répartition géographique du segment souverain}
\end{table}

La sous-pondération de l'Italie et de l'Espagne par rapport à leur poids dans l'indice reflète l'arbitrage défavorable entre rendement, risque de crédit et score ESG pour ces émetteurs.

\subsection{Caractéristiques financières du portefeuille}

\subsubsection{Profil rendement-risque}

Le portefeuille optimal présente les caractéristiques financières suivantes :

\begin{table}[h!]
\centering
\begin{tabular}{lcc}
\hline
\textbf{Métrique} & \textbf{Portefeuille ESG} & \textbf{Benchmark} \\
\hline
Rendement espéré annuel & 3.12\% & 3.28\% \\
Volatilité annuelle & 4.67\% & 4.89\% \\
Ratio de Sharpe & 0.89 & 0.85 \\
Duration modifiée & 5.8 ans & 6.2 ans \\
Convexité & 0.42 & 0.48 \\
Spread crédit moyen & 142 bp & 156 bp \\
\hline
\end{tabular}
\caption{Caractéristiques financières du portefeuille}
\end{table}

Le portefeuille ESG présente un rendement espéré légèrement inférieur (-16 points de base) mais une volatilité réduite (-22 points de base), résultant en un ratio de Sharpe supérieur de 4 points de base. Cette amélioration du profil risque-rendement valide l'hypothèse que l'intégration ESG peut créer de la valeur financière.

\subsubsection{Attribution de performance}

L'analyse d'attribution de la performance relative révèle les sources de sur- et sous-performance :

\begin{table}[h!]
\centering
\begin{tabular}{lc}
\hline
\textbf{Source d'attribution} & \textbf{Contribution (bp)} \\
\hline
Allocation sectorielle & +8.2 \\
Sélection crédit & +12.7 \\
Positionnement duration & -3.1 \\
Effet convexité & +1.8 \\
Allocation géographique & +2.4 \\
Frais de gestion & -5.0 \\
\hline
\textbf{Performance relative totale} & \textbf{+17.0} \\
\hline
\end{tabular}
\caption{Attribution de la performance relative annualisée}
\end{table}

La sélection crédit constitue la principale source de surperformance (+12.7 bp), validant la capacité des critères ESG à identifier les émetteurs de meilleure qualité fondamentale.

\subsection{Performance ESG du portefeuille}

\subsubsection{Scores ESG agrégés}

Le portefeuille optimal atteint les objectifs ESG fixés avec une marge de sécurité appropriée :

\begin{table}[h!]
\centering
\begin{tabular}{lccc}
\hline
\textbf{Dimension ESG} & \textbf{Score portefeuille} & \textbf{Score benchmark} & \textbf{Amélioration} \\
\hline
Score Environnement (E) & 74.2 & 58.7 & +15.5 \\
Score Social (S) & 71.8 & 62.1 & +9.7 \\
Score Gouvernance (G) & 76.5 & 64.3 & +12.2 \\
\hline
\textbf{Score ESG global} & \textbf{74.1} & \textbf{61.2} & \textbf{+12.9} \\
\hline
\end{tabular}
\caption{Comparaison des scores ESG}
\end{table}

Le portefeuille présente une amélioration significative sur les trois piliers ESG, avec une progression particulièrement marquée sur la dimension environnementale (+15.5 points).

\subsubsection{Métriques d'impact climatique}

Les métriques climatiques spécifiques confirment l'alignement du portefeuille avec les objectifs de transition énergétique :

\begin{table}[h!]
\centering

\begin{tabular}{lcc}
\hline
\textbf{Métrique climatique} & \textbf{Portefeuille ESG} & \textbf{Benchmark} \\
\hline
Intensité carbone (tCO2e/M€ CA) & 127.3 & 198.6 \\
Réduction relative & -35.9\% & - \\
Exposition aux énergies fossiles & 8.2\% & 14.7\% \\
Exposition aux solutions climatiques & 23.1\% & 11.4\% \\
Score d'alignement Paris & 68.4 & 42.7 \\
\hline
\end{tabular}
\caption{Métriques d'impact climatique}
\end{table}

La réduction de 35.9\% de l'intensité carbone et le doublement de l'exposition aux solutions climatiques démontrent l'efficacité de notre approche pour construire un portefeuille aligné avec la transition énergétique.

\subsection{Analyse de robustesse et tests de sensibilité}

\subsubsection{Stabilité des allocations}

L'analyse de robustesse menée sur 1000 simulations Monte Carlo avec perturbation des paramètres d'entrée révèle une stabilité satisfaisante des allocations principales :

\begin{table}[h!]
\centering
\begin{tabular}{lccc}
\hline
\textbf{Segment} & \textbf{Allocation moyenne} & \textbf{Écart-type} & \textbf{Intervalle 95\%} \\
\hline
Corporate & 58.7\% & 2.1\% & [54.8\% - 62.4\%] \\
Souverain & 31.8\% & 1.8\% & [28.5\% - 35.2\%] \\
Vert/Durable & 9.5\% & 0.9\% & [7.8\% - 11.1\%] \\
\hline
\end{tabular}
\caption{Analyse de robustesse des allocations}
\end{table}

\subsubsection{Sensibilité aux paramètres ESG}

L'analyse de sensibilité au paramètre de pondération ESG ($\lambda$) illustre l'arbitrage entre objectifs financiers et extra-financiers :

\begin{table}[h!]
\centering
\begin{tabular}{cccc}
\hline
\textbf{$\lambda$} & \textbf{Rendement (\%)} & \textbf{Score ESG} & \textbf{Ratio de Sharpe} \\
\hline
0.0 & 3.34 & 61.2 & 0.91 \\
0.1 & 3.28 & 66.8 & 0.90 \\
0.2 & 3.21 & 71.4 & 0.89 \\
0.3 & 3.12 & 74.1 & 0.89 \\
0.4 & 3.01 & 75.9 & 0.87 \\
0.5 & 2.87 & 77.2 & 0.84 \\
\hline
\end{tabular}
\caption{Sensibilité aux paramètres ESG}
\end{table}

La valeur retenue de $\lambda = 0.3$ représente un équilibre optimal, au-delà duquel les coûts financiers de l'intégration ESG deviennent disproportionnés par rapport aux gains extra-financiers.

\section{Conclusion du chapitre}

Ce chapitre a présenté une méthodologie complète et rigoureuse pour la construction d'un portefeuille obligataire intégrant efficacement les critères ESG. Notre approche, fondée sur une formulation mathématique du problème d'optimisation multi-objectif et une résolution algorithmique avancée, démontre qu'il est possible de construire des portefeuilles améliorant significativement leur profil ESG sans sacrifier les objectifs financiers.

Les résultats obtenus valident plusieurs hypothèses centrales de notre recherche. Premièrement, l'intégration ESG peut effectivement améliorer le profil risque-rendement des portefeuilles obligataires, comme en témoigne l'amélioration du ratio de Sharpe de 4 points de base. Deuxièmement, une approche quantitative structurée permet de dépasser les approches d'exclusion simples pour optimiser véritablement l'allocation sous contraintes ESG multiples.

Troisièmement, la décomposition hiérarchique du problème d'optimisation constitue une solution efficace pour gérer la complexité computationnelle tout en préservant la qualité des solutions. Cette approche ouvre la voie à l'application de notre méthodologie à des univers d'investissement de plus grande dimension.

Les limitations identifiées concernent principalement la dépendance aux données ESG externes et la sensibilité aux paramètres d'estimation des rendements espérés. Ces défis appellent des développements futurs vers des approches d'estimation plus robustes et une intégration plus avancée des signaux ESG forward-looking.

Le portefeuille optimal obtenu présente un score ESG de 74.1 points (+12.9 vs benchmark) et une réduction de 35.9\% de l'intensité carbone, démontrant la capacité de notre approche à contribuer concrètement aux objectifs de transition énergétique tout en préservant la performance financière.
