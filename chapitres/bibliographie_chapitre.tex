\chapter{Bibliographie}

\section{Références académiques et littérature spécialisée}

Cette bibliographie présente l'ensemble des sources académiques, réglementaires et professionnelles qui ont fondé cette recherche sur la modélisation du risque de crédit intégrant les critères ESG et les techniques de machine learning.

\subsection{Ouvrages de référence en finance quantitative}

\textbf{Altman, E. I.} (1968). \textit{Financial ratios, discriminant analysis and the prediction of corporate bankruptcy}. Journal of Finance, 23(4), 589-609.

\textbf{Black, F., \& Scholes, M.} (1973). \textit{The pricing of options and corporate liabilities}. Journal of Political Economy, 81(3), 637-654.

\textbf{Duffie, D., \& Singleton, K. J.} (2003). \textit{Credit Risk: Pricing, Measurement, and Management}. Princeton University Press.

\textbf{Hull, J. C.} (2018). \textit{Options, Futures, and Other Derivatives} (10th ed.). Pearson.

\textbf{Lando, D.} (2004). \textit{Credit Risk Modeling: Theory and Applications}. Princeton University Press.

\textbf{Markowitz, H.} (1952). \textit{Portfolio selection}. Journal of Finance, 7(1), 77-91.

\textbf{Merton, R. C.} (1974). \textit{On the pricing of corporate debt: The risk structure of interest rates}. Journal of Finance, 29(2), 449-470.

\textbf{Sharpe, W. F.} (1964). \textit{Capital asset prices: A theory of market equilibrium under conditions of risk}. Journal of Finance, 19(3), 425-442.

\subsection{Littérature sur le risque de crédit et modélisation}

\textbf{Altman, E. I., \& Saunders, A.} (1997). \textit{Credit risk measurement: Developments over the last 20 years}. Journal of Banking \& Finance, 21(11), 1721-1742.

\textbf{Arora, N., Bohn, J. R., \& Zhu, F.} (2005). \textit{Reduced form vs. structural models of credit risk: A case study of three models}. Journal of Investment Management, 3(4), 43-67.

\textbf{Bharath, S. T., \& Shumway, T.} (2008). \textit{Forecasting default with the Merton distance to default model}. Review of Financial Studies, 21(3), 1339-1369.

\textbf{Campbell, J. Y., Hilscher, J., \& Szilagyi, J.} (2008). \textit{In search of distress risk}. Journal of Finance, 63(6), 2899-2939.

\textbf{Das, S. R., Freed, L., Geng, G., \& Kapadia, N.} (2006). \textit{Correlated default risk}. Journal of Fixed Income, 16(2), 7-32.

\textbf{Duffie, D., Saita, L., \& Wang, K.} (2007). \textit{Multi-period corporate default prediction with stochastic covariates}. Journal of Financial Economics, 83(3), 635-665.

\textbf{Jarrow, R. A., \& Turnbull, S. M.} (1995). \textit{Pricing derivatives on financial securities subject to credit risk}. Journal of Finance, 50(1), 53-85.

\textbf{Kealhofer, S.} (2003). \textit{Quantifying credit risk I: Default prediction}. Financial Analysts Journal, 59(1), 30-44.

\textbf{Shumway, T.} (2001). \textit{Forecasting bankruptcy more accurately: A simple hazard model}. Journal of Business, 74(1), 101-124.

\textbf{Vassalou, M., \& Xing, Y.} (2004). \textit{Default risk in equity returns}. Journal of Finance, 59(2), 831-868.

\subsection{ESG et finance durable}

\textbf{Bauer, R., \& Hann, D.} (2010). \textit{Corporate environmental management and credit risk}. European Centre for Corporate Engagement Working Paper.

\textbf{Chava, S.} (2014). \textit{Environmental externalities and cost of capital}. Management Science, 60(9), 2223-2247.

\textbf{Crifo, P., Diaye, M. A., \& Oueghlissi, R.} (2017). \textit{The effect of countries' ESG ratings on their sovereign borrowing costs}. Quarterly Review of Economics and Finance, 66, 13-20.

\textbf{Drempetic, S., Klein, C., \& Zwergel, B.} (2020). \textit{The influence of firm size on the ESG score: Corporate sustainability ratings under review}. Journal of Business Ethics, 167(2), 333-360.

\textbf{Dunn, J., Fitzgibbons, S., \& Pomorski, L.} (2018). \textit{ESG integration and downside risk}. Journal of Portfolio Management, 44(5), 20-34.

\textbf{Edmans, A.} (2011). \textit{Does the stock market fully value intangibles? Employee satisfaction and equity prices}. Journal of Financial Economics, 101(3), 621-640.

\textbf{Friede, G., Busch, T., \& Bassen, A.} (2015). \textit{ESG and financial performance: Aggregated evidence from more than 2000 empirical studies}. Journal of Sustainable Finance \& Investment, 5(4), 210-233.

\textbf{Gillan, S. L., Koch, A., \& Starks, L. T.} (2021). \textit{Firms and social responsibility: A review of ESG and CSR research in corporate finance}. Journal of Corporate Finance, 66, 101889.

\textbf{Hong, H., \& Kacperczyk, M.} (2009). \textit{The price of sin: The effects of social norms on markets}. Journal of Financial Economics, 93(1), 15-36.

\textbf{Kölbel, J. F., Busch, T., \& Jancso, L. M.} (2017). \textit{How media coverage of corporate social irresponsibility increases financial risk}. Strategic Management Journal, 38(11), 2266-2284.

\textbf{Lins, K. V., Servaes, H., \& Tamayo, A.} (2017). \textit{Social capital, trust, and firm performance: The value of corporate social responsibility during the financial crisis}. Journal of Finance, 72(4), 1785-1824.

\textbf{Stellner, C., Klein, C., \& Zwergel, B.} (2015). \textit{Corporate social responsibility and Eurozone corporate bonds: The moderating role of country sustainability}. Journal of Banking \& Finance, 59, 538-549.

\subsection{Machine Learning appliqué à la finance}

\textbf{Bao, Y., Hilary, G., \& Ke, B.} (2022). \textit{Artificial intelligence and fraud detection}. Review of Accounting Studies, 27(1), 146-191.

\textbf{Breiman, L.} (2001). \textit{Random forests}. Machine Learning, 45(1), 5-32.

\textbf{Chen, T., \& Guestrin, C.} (2016). \textit{XGBoost: A scalable tree boosting system}. Proceedings of the 22nd ACM SIGKDD International Conference on Knowledge Discovery and Data Mining, 785-794.

\textbf{Feng, G., Giglio, S., \& Xiu, D.} (2020). \textit{Taming the factor zoo: A test of new factors}. Journal of Finance, 75(3), 1327-1370.

\textbf{Freyberger, J., Neuhierl, A., \& Weber, M.} (2020). \textit{Dissecting characteristics nonparametrically}. Review of Financial Studies, 33(5), 2326-2377.

\textbf{Gu, S., Kelly, B., \& Xiu, D.} (2020). \textit{Empirical asset pricing via machine learning}. Review of Financial Studies, 33(5), 2223-2273.

\textbf{Kozak, S., Nagel, S., \& Santosh, S.} (2020). \textit{Shrinking the cross-section}. Journal of Financial Economics, 135(2), 271-292.

\textbf{Lundberg, S. M., \& Lee, S. I.} (2017). \textit{A unified approach to interpreting model predictions}. Advances in Neural Information Processing Systems, 30, 4765-4774.

\textbf{Ribeiro, M. T., Singh, S., \& Guestrin, C.} (2016). \textit{"Why should I trust you?": Explaining the predictions of any classifier}. Proceedings of the 22nd ACM SIGKDD International Conference on Knowledge Discovery and Data Mining, 1135-1144.

\textbf{Sirignano, J., Sadhwani, A., \& Giesecke, K.} (2016). \textit{Deep learning for mortgage risk}. arXiv preprint arXiv:1607.02470.

\subsection{Machine Learning et risque de crédit}

\textbf{Addo, P. M., Guegan, D., \& Hassani, B.} (2018). \textit{Credit risk analysis using machine and deep learning models}. Risks, 6(2), 38.

\textbf{Barboza, F., Kimura, H., \& Altman, E.} (2017). \textit{Machine learning models and bankruptcy prediction}. Expert Systems with Applications, 83, 405-417.

\textbf{Butaru, F., Chen, Q., Clark, B., Das, S., Lo, A. W., \& Siddique, A.} (2016). \textit{Risk and risk management in the credit card industry}. Journal of Banking \& Finance, 72, 218-239.

\textbf{Khandani, A. E., Kim, A. J., \& Lo, A. W.} (2010). \textit{Consumer credit-risk models via machine-learning algorithms}. Journal of Banking \& Finance, 34(11), 2767-2777.

\textbf{Leo, M., Sharma, S., \& Maddulety, K.} (2019). \textit{Machine learning in banking risk management: A literature review}. Risks, 7(1), 29.

\textbf{Moscatelli, M., Narizzano, S., Parlapiano, F., \& Viggiano, G.} (2020). \textit{Corporate default forecasting with machine learning}. Expert Systems with Applications, 161, 113567.

\textbf{Serrano-Cinca, C., \& Gutiérrez-Nieto, B.} (2016). \textit{The use of profit scoring as an alternative to credit scoring systems in peer-to-peer (P2P) lending}. Decision Support Systems, 89, 113-122.

\textbf{Yao, X., Crook, J., \& Andreeva, G.} (2017). \textit{Deep learning in credit scoring: The need for explainable artificial intelligence}. Credit Research Centre Working Paper.

\subsection{Optimisation de portefeuille et allocation d'actifs}

\textbf{Black, F., \& Litterman, R.} (1992). \textit{Global portfolio optimization}. Financial Analysts Journal, 48(5), 28-43.

\textbf{DeMiguel, V., Garlappi, L., \& Uppal, R.} (2009). \textit{Optimal versus naive diversification: How inefficient is the 1/N portfolio strategy?}. Review of Financial Studies, 22(5), 1915-1953.

\textbf{Fabozzi, F. J., Kolm, P. N., Pachamanova, D. A., \& Focardi, S. M.} (2007). \textit{Robust Portfolio Optimization and Management}. John Wiley \& Sons.

\textbf{Jagannathan, R., \& Ma, T.} (2003). \textit{Risk reduction in large portfolios: Why imposing the wrong constraints helps}. Journal of Finance, 58(4), 1651-1683.

\textbf{Ledoit, O., \& Wolf, M.} (2004). \textit{A well-conditioned estimator for large-dimensional covariance matrices}. Journal of Multivariate Analysis, 88(2), 365-411.

\textbf{Meucci, A.} (2005). \textit{Risk and Asset Allocation}. Springer Finance.

\subsection{ESG et construction de portefeuille}

\textbf{Alessandrini, F., \& Jondeau, E.} (2020). \textit{ESG investing: From sin stocks to smart beta}. Journal of Portfolio Management, 46(3), 75-94.

\textbf{Berg, F., Kölbel, J. F., \& Rigobon, R.} (2022). \textit{Aggregate confusion: The divergence of ESG ratings}. Review of Finance, 26(6), 1315-1344.

\textbf{Cornell, B., \& Damodaran, A.} (2020). \textit{Valuing ESG: Doing good or sounding good?}. Journal of Impact and ESG Investing, 1(1), 76-93.

\textbf{Dimson, E., Marsh, P., \& Staunton, M.} (2020). \textit{Divergent ESG ratings}. Journal of Portfolio Management, 47(1), 75-87.

\textbf{Pedersen, L. H., Fitzgibbons, S., \& Pomorski, L.} (2021). \textit{Responsible investing: The ESG-efficient frontier}. Journal of Financial Economics, 142(2), 572-597.

\textbf{Riedl, A., \& Smeets, P.} (2017). \textit{Why do investors hold socially responsible mutual funds?}. Journal of Finance, 72(6), 2505-2550.

\subsection{Méthodes quantitatives et évaluation de modèles}

\textbf{DeLong, E. R., DeLong, D. M., \& Clarke-Pearson, D. L.} (1988). \textit{Comparing the areas under two or more correlated receiver operating characteristic curves: A nonparametric approach}. Biometrics, 44(3), 837-845.

\textbf{Fawcett, T.} (2006). \textit{An introduction to ROC analysis}. Pattern Recognition Letters, 27(8), 861-874.

\textbf{Hand, D. J.} (2009). \textit{Measuring classifier performance: A coherent alternative to the area under the ROC curve}. Machine Learning, 77(1), 103-123.

\textbf{Hastie, T., Tibshirani, R., \& Friedman, J.} (2009). \textit{The Elements of Statistical Learning: Data Mining, Inference, and Prediction} (2nd ed.). Springer.

\textbf{James, G., Witten, D., Hastie, T., \& Tibshirani, R.} (2013). \textit{An Introduction to Statistical Learning}. Springer.

\section{Documents réglementaires et institutionnels}

\subsection{Réglementation bancaire et financière}

\textbf{Basel Committee on Banking Supervision} (2017). \textit{Basel III: Finalising post-crisis reforms}. Bank for International Settlements.

\textbf{Basel Committee on Banking Supervision} (2021). \textit{Climate-related risk drivers and their transmission channels}. Bank for International Settlements.

\textbf{Commission européenne} (2020). \textit{Règlement (UE) 2020/852 sur l'établissement d'un cadre visant à favoriser les investissements durables} (Taxonomie).

\textbf{Commission européenne} (2019). \textit{Règlement (UE) 2019/2088 sur la publication d'informations en matière de durabilité dans le secteur des services financiers} (SFDR).

\textbf{Autorité bancaire européenne} (2021). \textit{Guidelines on loan origination and monitoring}. EBA/GL/2020/06.

\textbf{Autorité européenne des assurances et des pensions professionnelles} (2019). \textit{Opinion on sustainability within Solvency II}. EIOPA-BoS-19/241.

\subsection{Supervision et cadres prudentiels}

\textbf{Banque centrale européenne} (2020). \textit{Guide on climate-related and environmental risks: Supervisory expectations relating to risk management and disclosure}. BCE.

\textbf{Federal Reserve Board} (2021). \textit{Supervisory guidance on model risk management}. SR 11-7.

\textbf{Office of the Comptroller of the Currency} (2021). \textit{Principles for climate-related financial risk management for large banks}. OCC Bulletin 2021-62.

\textbf{Prudential Regulation Authority} (2021). \textit{Climate change adaptation report 2021}. Bank of England.

\subsection{Organisations internationales}

\textbf{Financial Stability Board} (2020). \textit{The implications of climate change for financial stability}. FSB.

\textbf{Network for Greening the Financial System} (2021). \textit{Scenarios for central banks and supervisors}. NGFS Technical Document.

\textbf{Task Force on Climate-related Financial Disclosures} (2017). \textit{Recommendations of the Task Force on Climate-related Financial Disclosures}. TCFD.

\textbf{International Organization of Securities Commissions} (2021). \textit{Sustainable finance and the role of securities regulators and IOSCO}. IOSCO Final Report.

\section{Rapports et études sectorielles}

\subsection{Agences de notation et fournisseurs de données}

\textbf{Moody's Investors Service} (2021). \textit{Environmental, social and governance risks increasingly influence credit outcomes}. Moody's ESG Solutions.

\textbf{Standard \& Poor's Global Ratings} (2020). \textit{The ESG risk atlas: Sector and regional rationales and scores}. S\&P Global.

\textbf{Fitch Ratings} (2021). \textit{ESG in credit: From niche to mainstream}. Fitch ESG Solutions.

\textbf{MSCI Inc.} (2020). \textit{ESG and the cost of capital}. MSCI Research Report.

\textbf{Sustainalytics} (2021). \textit{ESG risk ratings methodology abstract}. Morningstar Sustainalytics.

\subsection{Institutions financières et gestionnaires d'actifs}

\textbf{BlackRock Investment Institute} (2021). \textit{Climate change and the global economy: Macro implications for monetary policy, financial stability, and the economy}. BlackRock.

\textbf{Goldman Sachs Asset Management} (2020). \textit{ESG engagement: Transforming companies for the future}. Goldman Sachs.

\textbf{JP Morgan Asset Management} (2021). \textit{Why ESG integration is evolving beyond exclusions}. JP Morgan.

\textbf{Vanguard} (2020). \textit{ESG integration in active fundamental equity}. Vanguard Research.

\subsection{Études académiques et think tanks}

\textbf{Centre for European Policy Studies} (2021). \textit{Sustainable finance: The EU taxonomy}. CEPS Policy Insights.

\textbf{Institute of International Finance} (2020). \textit{ESG in emerging markets: External financing, local ownership}. IIF Working Group Report.

\textbf{McKinsey Global Institute} (2020). \textit{Climate risk and response: Physical hazards and socioeconomic impacts}. McKinsey \& Company.

\textbf{World Economic Forum} (2021). \textit{Measuring stakeholder capitalism: Towards common metrics and consistent reporting of sustainable value creation}. WEF.

\section{Sources de données et méthodologies}

\subsection{Bases de données financières}

\textbf{Bloomberg Terminal} - Données de marché obligataire, spreads de crédit, notations, et indicateurs macroéconomiques.

\textbf{Refinitiv Eikon} - Données financières fondamentales, prix des Credit Default Swaps, et historiques de défaut.

\textbf{FactSet} - Données sectorielles, comparables transactionnels, et métriques de valorisation.

\subsection{Fournisseurs de données ESG}

\textbf{MSCI ESG Research} - Scores ESG détaillés, analyse des controverses, et métriques climatiques.

\textbf{Sustainalytics} - Notations de risque ESG, évaluations sectorielles, et analyse d'impact.

\textbf{S\&P Trucost} - Données environnementales, empreinte carbone, et analyse des risques physiques.

\textbf{CDP (Carbon Disclosure Project)} - Données d'émissions auto-déclarées et stratégies climatiques des entreprises.

\subsection{Outils et plateformes technologiques}

\textbf{Python Software Foundation} - Langage de programmation pour l'implémentation des modèles de machine learning.

\textbf{Scikit-learn} - Bibliothèque d'apprentissage automatique pour Python.

\textbf{XGBoost Development Team} - Framework de gradient boosting optimisé.

\textbf{SHAP (SHapley Additive exPlanations)} - Outils d'interprétabilité des modèles de machine learning.

Cette bibliographie représente l'ensemble des sources qui ont nourri cette recherche, combinant fondements théoriques solides, innovations méthodologiques récentes, et applications pratiques dans le domaine de l'intersection entre critères ESG, techniques de machine learning, et évaluation du risque de crédit obligataire.
