\chapter*{Résumé}
\addcontentsline{toc}{chapter}{Résumé}

Cette recherche examine l'intégration des critères Environnementaux, Sociaux et de Gouvernance (ESG) dans la modélisation du risque de crédit des portefeuilles obligataires, en explorant la valeur ajoutée des techniques de Machine Learning par rapport aux approches traditionnelles.

L'objectif principal consiste à démontrer que l'utilisation conjointe des critères ESG et des techniques de Machine Learning peut améliorer significativement la précision des modèles de risque de crédit, tout en optimisant simultanément la performance financière et extra-financière des portefeuilles obligataires.

La méthodologie s'articule autour de trois axes. Premièrement, constitution d'un univers d'investissement de 847 obligations réparti entre obligations corporates (60%), souveraines (30%) et vertes (10%). Deuxièmement, implémentation et comparaison de neuf modèles de Machine Learning avec validation croisée temporelle sur douze mois. Troisièmement, conception d'un problème d'optimisation multi-objectif intégrant explicitement les contraintes ESG.

Les résultats démontrent la supériorité des modèles de Machine Learning intégrant les facteurs ESG. Le modèle de stacking atteint un AUC-ROC de 0.889, soit une amélioration de 14.8 points par rapport à la régression logistique (0.741). La capacité d'anticipation progresse significativement avec un Early Warning Score de 3.8 mois contre 1.9 mois pour les approches classiques. Les facteurs ESG contribuent collectivement à 13.1% de la performance prédictive, avec prédominance de la gouvernance (42%), suivie des facteurs environnementaux (35%) et sociaux (23%).

L'application à la construction de portefeuille confirme la valeur pratique. Le portefeuille optimal présente un score ESG de 74.1 points (+12.9 vs benchmark) tout en maintenant un ratio de Sharpe compétitif de 0.89. Cette performance s'accompagne d'une réduction de l'empreinte carbone de 35.9% et d'un doublement de l'exposition aux solutions climatiques.

Les tests de robustesse révèlent une stabilité temporelle supérieure des modèles avancés, avec un Temporal Stability Ratio de 2.68 pour le stacking contre des dégradations limitées à 8.9% dans les scénarios climatiques adverses. L'écart de performance est particulièrement marqué pour les émetteurs High Yield (+11.3% en AUC-ROC).

L'analyse coût-bénéfice indique un seuil de rentabilité d'environ 350-400 millions d'euros d'actifs sous gestion. L'intégration opérationnelle nécessite une architecture décisionnelle à trois niveaux : stratégique, tactique et opérationnel.

Cette recherche quantifie précisément l'apport des facteurs ESG à la prédiction du risque de crédit et démontre que l'investissement responsable peut créer de la valeur financière through une meilleure évaluation du risque. Les limites concernent principalement la dépendance aux données ESG externes et leur standardisation imparfaite.

En conclusion, l'intégration des critères ESG via des techniques de Machine Learning avancées constitue une approche prometteuse pour améliorer simultanément la gestion des risques et l'impact extra-financier des portefeuilles obligataires, contribuant à la transformation de la finance vers un modèle plus durable sans compromettre les objectifs de performance.

\textbf{Mots-clés :} Finance durable, Risque de crédit, Machine Learning, Portefeuille obligataire, Critères ESG, Optimisation multi-objectif, Modèles d'ensemble, Gestion des risques