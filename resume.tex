\chapter*{Résumé}
\addcontentsline{toc}{chapter}{Résumé}

Ce mémoire propose une analyse approfondie de l'intégration des critères Environnementaux, Sociaux et de Gouvernance (ESG) dans la modélisation du risque de crédit des portefeuilles obligataires, en explorant particulièrement la valeur ajoutée des techniques de Machine Learning par rapport aux approches traditionnelles. Dans un contexte où les investisseurs institutionnels doivent concilier performance financière et objectifs de durabilité, cette recherche développe un cadre méthodologique innovant pour optimiser cette intégration.

La problématique centrale de cette recherche réside dans l'identification des méthodes les plus efficaces pour incorporer les facteurs ESG dans l'évaluation du risque de crédit, tout en quantifiant précisément leur contribution à la capacité prédictive des modèles. L'objectif principal consiste à démontrer que l'utilisation conjointe des critères ESG et des techniques de Machine Learning peut améliorer significativement la précision des modèles de risque de crédit, tout en optimisant simultanément la performance financière et extra-financière des portefeuilles obligataires.

La méthodologie développée s'articule autour de trois axes principaux. Premièrement, nous avons constitué un univers d'investissement de 847 obligations sélectionnées selon des critères stricts de liquidité, de qualité de crédit et de disponibilité des données ESG, réparti entre obligations corporates (60%), souveraines (30%) et vertes ou durables (10%). Deuxièmement, nous avons implémenté et comparé neuf modèles de Machine Learning, depuis la régression logistique traditionnelle jusqu'aux approches d'ensemble les plus sophistiquées, en utilisant une validation croisée temporelle rigoureuse avec un horizon de prédiction de douze mois. Troisièmement, nous avons conçu un problème d'optimisation multi-objectif pour la construction de portefeuille, intégrant explicitement les contraintes ESG dans la fonction d'utilité.

Les résultats empiriques démontrent la supériorité significative des modèles de Machine Learning intégrant les facteurs ESG. Le modèle de stacking, combinant plusieurs algorithmes de base, atteint un AUC-ROC de 0.889, soit une amélioration de 14.8 points par rapport à la régression logistique traditionnelle (0.741). Cette supériorité se manifeste particulièrement dans la capacité d'anticipation, avec un Early Warning Score de 3.8 mois contre 1.9 mois pour les approches classiques. L'analyse de la contribution ESG révèle que les facteurs environnementaux, sociaux et de gouvernance apportent collectivement 13.1% de la performance prédictive totale, avec une prédominance des critères de gouvernance (42% de la contribution ESG), suivis des facteurs environnementaux (35%) et sociaux (23%).

L'application à la construction de portefeuille confirme la valeur pratique de cette approche. Le portefeuille optimal obtenu présente un score ESG de 74.1 points, soit une amélioration de 12.9 points par rapport au benchmark, tout en maintenant un ratio de Sharpe compétitif de 0.89. Cette performance s'accompagne d'une réduction substantielle de l'empreinte carbone (-35.9%) et d'un doublement de l'exposition aux solutions climatiques, démontrant la capacité de l'approche à contribuer concrètement aux objectifs de transition énergétique.

Les tests de robustesse révèlent une stabilité temporelle supérieure des modèles de Machine Learning face aux variations de conditions de marché. Le Temporal Stability Ratio du modèle de stacking (2.68) indique une résilience significativement plus élevée que les approches traditionnelles, particulièrement précieuse lors des périodes de stress économique. Les scénarios de transition climatique confirment cette robustesse, avec une dégradation limitée à 8.9% dans le pire cas contre 14.2% pour la régression logistique.

L'analyse comparative entre modèles traditionnels et Machine Learning révèle des écarts de performance particulièrement marqués pour les émetteurs High Yield, où l'amélioration atteint 11.3% en AUC-ROC contre 7.6% pour l'Investment Grade. Cette différenciation suggère que la complexité des relations ESG-crédit justifie d'autant plus l'utilisation d'approches avancées que le profil de risque des émetteurs est élevé.

Les implications pour la gestion obligataire sont multiples. L'intégration opérationnelle des modèles avancés nécessite une architecture décisionnelle à trois niveaux : stratégique pour l'allocation d'actifs, tactique pour la sélection d'émetteurs, et opérationnel pour l'exécution et le suivi. Cette structuration permet une incorporation cohérente des signaux de risque à différentes échelles temporelles. L'analyse coût-bénéfice indique un seuil de rentabilité d'environ 350-400 millions d'euros d'actifs sous gestion pour justifier l'investissement dans ces approches sophistiquées.

Cette recherche apporte plusieurs contributions originales à la littérature académique et aux pratiques professionnelles. Sur le plan théorique, elle quantifie précisément l'apport des facteurs ESG à la prédiction du risque de crédit et démontre les avantages des techniques d'ensemble pour capturer les interactions complexes entre variables financières et extra-financières. Sur le plan méthodologique, elle propose un cadre d'optimisation multi-objectif opérationnel pour la construction de portefeuilles obligataires sous contraintes ESG. Sur le plan empirique, elle valide l'hypothèse que l'investissement responsable peut créer de la valeur financière à travers une meilleure évaluation du risque.

Les limites identifiées concernent principalement la dépendance aux données ESG externes, dont l'historique limité et la standardisation imparfaite peuvent affecter la robustesse des modèles. Les perspectives d'amélioration incluent l'intégration de données alternatives (satellitaires, textuelles), le développement de modèles explicables plus sophistiqués, et l'extension à d'autres classes d'actifs.

En conclusion, cette recherche démontre que l'intégration des critères ESG via des techniques de Machine Learning avancées constitue une approche prometteuse pour améliorer simultanément la gestion des risques et l'impact extra-financier des portefeuilles obligataires. Les résultats obtenus suggèrent que cette voie méthodologique peut contribuer significativement à la transformation de la finance vers un modèle plus durable, sans compromettre les objectifs de performance traditionnels.

\textbf{Mots-clés :} Finance durable, Risque de crédit, Machine Learning, Portefeuille obligataire, Critères ESG, Optimisation multi-objectif, Modèles d'ensemble, Gestion des risques