\chapter{Annexes}

\section{Annexe A : Formulations mathématiques détaillées}

\subsection{Modèle de Merton (1974)}

Le modèle de Merton établit la relation entre la valeur des actifs d'une entreprise et la probabilité de défaut. La valeur des actifs suit un processus de mouvement brownien géométrique :

\begin{equation}
dV_t = \mu V_t dt + \sigma V_t dW_t
\end{equation}

où $V_t$ représente la valeur des actifs au temps $t$, $\mu$ le taux de dérive, $\sigma$ la volatilité des actifs, et $dW_t$ un processus de Wiener.

La probabilité de défaut sur l'horizon $T$ est donnée par :

\begin{equation}
PD = N\left(\frac{\ln(D/V_0) - (\mu - \frac{\sigma^2}{2})T}{\sigma\sqrt{T}}\right)
\end{equation}

où $D$ représente la dette de l'entreprise, $V_0$ la valeur initiale des actifs, et $N(\cdot)$ la fonction de répartition de la loi normale standard.

\subsection{Optimisation de portefeuille avec contraintes ESG}

La formulation du problème d'optimisation multi-objectif utilisée dans cette recherche s'écrit :

\begin{align}
\max_{\mathbf{w}} \quad & U(\mathbf{w}) = (1-\lambda) \cdot U_{fin}(\mathbf{w}) + \lambda \cdot U_{ESG}(\mathbf{w}) \\
\text{s.t.} \quad & \mathbf{w}^T \mathbf{1} = 1 \\
& \mathbf{w} \geq \mathbf{0} \\
& \mathbf{C w} \leq \mathbf{b}
\end{align}

où la fonction d'utilité financière est définie comme :

\begin{equation}
U_{fin}(\mathbf{w}) = \frac{\mathbf{w}^T \boldsymbol{\mu} - r_f}{\sqrt{\mathbf{w}^T \boldsymbol{\Sigma} \mathbf{w}}}
\end{equation}

et la fonction d'utilité ESG comme :

\begin{equation}
U_{ESG}(\mathbf{w}) = \frac{\mathbf{w}^T \mathbf{s}_{ESG} - s_{min}}{s_{max} - s_{min}}
\end{equation}

\section{Annexe B : Résultats détaillés des modèles de Machine Learning}

\subsection{Performance comparative des modèles}

Les résultats de l'évaluation comparative des modèles de Machine Learning sur données crédit-ESG sont présentés ci-dessous, utilisant la validation croisée temporelle avec un horizon de prédiction de 12 mois :

\begin{table}[h]
\centering
\caption{Performance comparative des modèles de machine learning}
\begin{tabular}{lccccc}
\toprule
\textbf{Modèle} & \textbf{AUC-ROC} & \textbf{AUC-PR} & \textbf{Brier Score} & \textbf{EWS (mois)} & \textbf{ECI (\%)} \\
\midrule
Régression Logistique & 0.741 ± 0.018 & 0.382 ± 0.021 & 0.142 ± 0.007 & 1.9 ± 0.3 & 3.2 ± 0.8 \\
Random Forest & 0.823 ± 0.015 & 0.452 ± 0.019 & 0.118 ± 0.006 & 2.6 ± 0.4 & 8.7 ± 1.2 \\
Gradient Boosting & 0.847 ± 0.013 & 0.487 ± 0.017 & 0.104 ± 0.005 & 2.9 ± 0.3 & 9.8 ± 1.1 \\
XGBoost & 0.872 ± 0.011 & 0.526 ± 0.015 & 0.097 ± 0.004 & 3.4 ± 0.3 & 11.2 ± 0.9 \\
LightGBM & 0.868 ± 0.012 & 0.517 ± 0.016 & 0.099 ± 0.005 & 3.3 ± 0.3 & 10.8 ± 1.0 \\
Réseau Neuronal (MLP) & 0.853 ± 0.014 & 0.494 ± 0.018 & 0.101 ± 0.006 & 3.1 ± 0.4 & 9.3 ± 1.3 \\
Réseau Récurrent (LSTM) & 0.861 ± 0.013 & 0.508 ± 0.017 & 0.098 ± 0.005 & 3.6 ± 0.4 & 10.5 ± 1.2 \\
Ensemble Modulaire & 0.878 ± 0.010 & 0.535 ± 0.014 & 0.093 ± 0.004 & 3.5 ± 0.3 & 12.4 ± 0.8 \\
Stacking & \textbf{0.889 ± 0.009} & \textbf{0.549 ± 0.013} & \textbf{0.091 ± 0.004} & \textbf{3.8 ± 0.2} & \textbf{13.1 ± 0.7} \\
\bottomrule
\end{tabular}
\end{table}

\subsection{Robustesse temporelle des modèles}

L'analyse de robustesse temporelle évalue les performances des modèles sur trois sous-périodes distinctes :

\begin{table}[h]
\centering
\caption{Robustesse temporelle des modèles (AUC-ROC)}
\begin{tabular}{lccccc}
\toprule
\textbf{Modèle} & \textbf{Stabilité} & \textbf{Stress} & \textbf{Reprise} & \textbf{Écart Max} & \textbf{TSR} \\
& \textbf{(2019-2020)} & \textbf{(2020-2021)} & \textbf{(2021-2023)} & & \\
\midrule
Régression Logistique & 0.763 & 0.712 & 0.741 & 0.051 & 1.00 \\
Random Forest & 0.835 & 0.798 & 0.819 & 0.037 & 1.38 \\
XGBoost & 0.883 & 0.857 & 0.871 & 0.026 & 1.96 \\
Réseau Neuronal (MLP) & 0.865 & 0.832 & 0.849 & 0.033 & 1.55 \\
Ensemble Modulaire & 0.889 & 0.865 & 0.877 & 0.024 & 2.13 \\
Stacking & \textbf{0.898} & \textbf{0.879} & \textbf{0.888} & \textbf{0.019} & \textbf{2.68} \\
\bottomrule
\end{tabular}
\end{table}

\subsection{Contribution des facteurs ESG par pilier}

La décomposition de l'indice de contribution ESG (ECI) par pilier révèle la contribution relative de chaque dimension :

\begin{table}[h]
\centering
\caption{Contribution relative des piliers ESG}
\begin{tabular}{lc}
\toprule
\textbf{Pilier ESG} & \textbf{Contribution relative (\%)} \\
\midrule
Gouvernance (G) & 42\% \\
Environnement (E) & 35\% \\
Social (S) & 23\% \\
\bottomrule
\end{tabular}
\end{table}

\section{Annexe C : Données du portefeuille obligataire}

\subsection{Constitution de l'univers d'investissement}

L'univers d'investissement comprend 847 obligations sélectionnées selon des critères stricts de liquidité, qualité de crédit et disponibilité des données ESG.

\begin{table}[h]
\centering
\caption{Composition de l'univers obligataire}
\begin{tabular}{lccc}
\toprule
\textbf{Type d'obligation} & \textbf{Nombre de titres} & \textbf{Allocation cible} & \textbf{Allocation réalisée} \\
\midrule
Obligations corporates & 508 & 60.0\% & 58.7\% \\
Obligations souveraines & 254 & 30.0\% & 31.8\% \\
Obligations vertes/durables & 85 & 10.0\% & 9.5\% \\
\midrule
\textbf{Total} & \textbf{847} & \textbf{100.0\%} & \textbf{100.0\%} \\
\bottomrule
\end{tabular}
\end{table}

\subsection{Caractéristiques financières du portefeuille optimal}

\begin{table}[h]
\centering
\caption{Profil rendement-risque du portefeuille}
\begin{tabular}{lcc}
\toprule
\textbf{Métrique} & \textbf{Portefeuille ESG} & \textbf{Benchmark} \\
\midrule
Rendement espéré annuel & 3.12\% & 3.28\% \\
Volatilité annuelle & 4.67\% & 4.89\% \\
Ratio de Sharpe & 0.89 & 0.85 \\
Duration modifiée & 5.8 ans & 6.2 ans \\
Convexité & 0.42 & 0.48 \\
Spread crédit moyen & 142 bp & 156 bp \\
\bottomrule
\end{tabular}
\end{table}

\subsection{Performance ESG du portefeuille}

\begin{table}[h]
\centering
\caption{Scores ESG du portefeuille optimal}
\begin{tabular}{lccc}
\toprule
\textbf{Dimension ESG} & \textbf{Score portefeuille} & \textbf{Score benchmark} & \textbf{Amélioration} \\
\midrule
Score Environnement (E) & 74.2 & 58.7 & +15.5 \\
Score Social (S) & 71.8 & 62.1 & +9.7 \\
Score Gouvernance (G) & 76.5 & 64.3 & +12.2 \\
\midrule
\textbf{Score ESG global} & \textbf{74.1} & \textbf{61.2} & \textbf{+12.9} \\
\bottomrule
\end{tabular}
\end{table}

\subsection{Métriques d'impact climatique}

\begin{table}[h]
\centering
\caption{Impact climatique du portefeuille}
\begin{tabular}{lcc}
\toprule
\textbf{Métrique climatique} & \textbf{Portefeuille ESG} & \textbf{Benchmark} \\
\midrule
Intensité carbone (tCO2e/M€ CA) & 127.3 & 198.6 \\
Réduction relative & -35.9\% & - \\
Exposition aux énergies fossiles & 8.2\% & 14.7\% \\
Exposition aux solutions climatiques & 23.1\% & 11.4\% \\
Score d'alignement Paris & 68.4 & 42.7 \\
\bottomrule
\end{tabular}
\end{table}

\section{Annexe D : Tests de robustesse et analyses de sensibilité}

\subsection{Résilience face aux scénarios de transition climatique}

Les tests de stress climatique évaluent la dégradation de performance des modèles sous différents scénarios de transition énergétique :

\begin{table}[h]
\centering
\caption{Impact des scénarios climatiques sur la performance des modèles (\% de dégradation)}
\begin{tabular}{lccc}
\toprule
\textbf{Modèle} & \textbf{Transition ordonnée} & \textbf{Transition désordonnée} & \textbf{Transition retardée} \\
\midrule
Régression Logistique & -5.3\% & -12.7\% & -14.2\% \\
Random Forest & -4.1\% & -9.8\% & -11.5\% \\
XGBoost & -3.6\% & -8.3\% & -9.7\% \\
Ensemble Modulaire & -3.8\% & -7.9\% & -10.1\% \\
Stacking & \textbf{-3.2\%} & \textbf{-7.5\%} & \textbf{-8.9\%} \\
\bottomrule
\end{tabular}
\end{table}

\subsection{Analyse de sensibilité aux paramètres ESG}

L'analyse de sensibilité au paramètre de pondération ESG ($\lambda$) illustre l'arbitrage entre objectifs financiers et extra-financiers :

\begin{table}[h]
\centering
\caption{Sensibilité du portefeuille aux paramètres ESG}
\begin{tabular}{cccc}
\toprule
\textbf{$\lambda$} & \textbf{Rendement (\%)} & \textbf{Score ESG} & \textbf{Ratio de Sharpe} \\
\midrule
0.0 & 3.34 & 61.2 & 0.91 \\
0.1 & 3.28 & 66.8 & 0.90 \\
0.2 & 3.21 & 71.4 & 0.89 \\
0.3 & 3.12 & 74.1 & 0.89 \\
0.4 & 3.01 & 75.9 & 0.87 \\
0.5 & 2.87 & 77.2 & 0.84 \\
\bottomrule
\end{tabular}
\end{table}

\section{Annexe E : Métriques d'évaluation spécialisées}

\subsection{Métriques adaptées au contexte financier}

En complément des métriques statistiques standards, plusieurs mesures spécialisées ont été développées :

\paragraph{Early Warning Score (EWS)}
Mesure la capacité du modèle à détecter précocement les dégradations de crédit :

\begin{equation}
EWS = \frac{1}{|E|} \sum_{i \in E} (t_{event,i} - t_{signal,i})
\end{equation}

\paragraph{ESG Contribution Index (ECI)}
Quantifie la contribution relative des variables ESG à la performance prédictive :

\begin{equation}
ECI = \frac{Perf(M_{full}) - Perf(M_{non\_ESG})}{Perf(M_{full}) - Perf(M_{random})}
\end{equation}

\paragraph{Temporal Stability Ratio (TSR)}
Évalue la stabilité temporelle des prédictions face aux variations de conditions de marché :

\begin{equation}
TSR = \frac{\sigma_{perf}(M_{benchmark})}{\sigma_{perf}(M_{evaluated})}
\end{equation}

\subsection{Métriques de calibration}

\paragraph{Brier Score}
Mesure quadratique de l'écart entre probabilités prédites et résultats observés :

\begin{equation}
BS = \frac{1}{n} \sum_{i=1}^{n} (f(x_i) - y_i)^2
\end{equation}

\paragraph{Expected Calibration Error (ECE)}
Mesure plus granulaire de la calibration, calculée en regroupant les prédictions en bins :

\begin{equation}
ECE = \sum_{j=1}^{M} \frac{|B_j|}{n} |\text{acc}(B_j) - \text{conf}(B_j)|
\end{equation}

\section{Annexe F : Algorithmes et pseudo-code}

\subsection{Validation croisée temporelle}

\begin{verbatim}
ALGORITHME ValidationCroiseeTemporelle
ENTRÉE: données, modèle, fenêtre_initiale = 36 mois, horizon = 12 mois
SORTIE: scores_validation

1. Ordonner chronologiquement les données
2. w = fenêtre_initiale, h = horizon de prédiction, s = 3 mois
3. POUR t = w à T-h par pas de s FAIRE
4.    Entraîner modèle sur [1:t]
5.    Évaluer modèle sur [t+1:t+h]
6.    Stocker métriques de performance
7. FIN POUR
8. Agréger les métriques (moyenne, écart-type)
9. RETOURNER scores_validation
\end{verbatim}

\subsection{Optimisation de portefeuille avec contraintes ESG}

\begin{verbatim}
ALGORITHME OptimisationPortefeuilleESG
ENTRÉE: rendements, scores_ESG, contraintes, lambda = 0.3
SORTIE: poids_optimaux

1. Initialiser les poids uniformément
2. POUR chaque itération FAIRE
3.    Calculer fonction objectif hybride:
       U = (1-λ) × Sharpe_ratio + λ × Score_ESG_normalisé
4.    Vérifier contraintes (diversification, duration, ESG_min)
5.    Ajuster poids selon gradient de la fonction objectif
6.    Vérifier convergence
7. FIN POUR
8. RETOURNER poids_optimaux
\end{verbatim}

\section{Annexe G : Sources et traitement des données}

\subsection{Sources de données}

\paragraph{Données financières}
Les données de marché proviennent de Bloomberg Terminal avec mise à jour quotidienne, incluant :
\begin{itemize}
\item Prix et rendements (cours de clôture, rendement à l'échéance, spreads)
\item Caractéristiques techniques (duration modifiée, convexité, sensibilité aux spreads)
\item Données d'émission (nominal, dates, taux de coupon, fréquence)
\item Notations (Moody's, S\&P, Fitch avec réconciliation numérique)
\end{itemize}

\paragraph{Données ESG}
Les scores ESG proviennent principalement de MSCI ESG Research, complétés par Sustainalytics pour validation croisée. L'agrégation combine :
\begin{itemize}
\item Score E (40\%) : intensité carbone, gestion déchets, efficacité énergétique
\item Score S (30\%) : relations sociales, sécurité produits, capital humain
\item Score G (30\%) : conseil d'administration, transparence, éthique
\end{itemize}

\subsection{Critères de sélection des titres}

Les obligations incluses dans l'univers répondent aux critères suivants :
\begin{itemize}
\item Liquidité minimale : volume quotidien > 10 millions d'euros (12 mois)
\item Maturité résiduelle : 1 à 30 ans
\item Notation minimale : BBB- ou équivalent
\item Devise : euros uniquement
\item Disponibilité ESG : score complet et récent (< 12 mois)
\end{itemize}

\section{Annexe H : Limites et perspectives}

\subsection{Limites méthodologiques identifiées}

\paragraph{Données ESG}
\begin{itemize}
\item Historique limité des données ESG standardisées (généralement post-2015)
\item Divergences méthodologiques entre fournisseurs de données
\item Couverture inégale selon les émetteurs et marchés
\item Biais potentiel de déclaration volontaire
\end{itemize}

\paragraph{Modélisation}
\begin{itemize}
\item Risque de surapprentissage pour les modèles complexes
\item Sensibilité aux paramètres d'estimation des rendements espérés
\item Hypothèses de stationnarité des relations ESG-crédit
\item Corrélations instables en période de crise
\end{itemize}

\subsection{Perspectives d'amélioration}

\paragraph{Données alternatives}
\begin{itemize}
\item Intégration de données satellitaires pour mesures environnementales directes
\item Analyse textuelle des rapports annuels et communications
\item Données comportementales et d'actualité en temps réel
\item Intelligence artificielle appliquée aux sources non structurées
\end{itemize}

\paragraph{Modèles avancés}
\begin{itemize}
\item Graph Neural Networks pour relations complexes entre émetteurs
\item Transformer Models pour analyse de séries temporelles financières
\item Approches d'inférence causale pour relations ESG-performance
\item Modèles explicables combinant performance et interprétabilité
\end{itemize}

\section{Annexe I : Glossaire technique}

\textbf{AUC-ROC} : Aire sous la courbe ROC, mesure de discrimination pour classification binaire.

\textbf{Duration modifiée} : Sensibilité approximative du prix d'une obligation aux variations de taux d'intérêt.

\textbf{Early Warning Score (EWS)} : Délai moyen entre signal d'alerte et événement de crédit.

\textbf{ESG Contribution Index (ECI)} : Contribution relative des variables ESG à la performance prédictive.

\textbf{Expected Shortfall} : Perte moyenne attendue au-delà du seuil de Value-at-Risk.

\textbf{Stacking} : Technique d'ensemble combinant plusieurs modèles via un méta-modèle.

\textbf{Temporal Stability Ratio (TSR)} : Mesure de stabilité temporelle des performances de modèles.

\textbf{Value-at-Risk (VaR)} : Perte maximale avec un niveau de confiance donné sur un horizon défini.