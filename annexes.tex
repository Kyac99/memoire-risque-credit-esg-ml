\chapter{Annexes}

\section{Annexe A : Formulations mathématiques détaillées}

\subsection{Modèle de Merton (1974)}

Le modèle de Merton établit la relation entre la valeur des actifs d'une entreprise et la probabilité de défaut. La valeur des actifs suit un processus de mouvement brownien géométrique :

\begin{equation}
dV_t = \mu V_t dt + \sigma V_t dW_t
\end{equation}

où $V_t$ représente la valeur des actifs au temps $t$, $\mu$ le taux de dérive, $\sigma$ la volatilité des actifs, et $dW_t$ un processus de Wiener.

La probabilité de défaut sur l'horizon $T$ est donnée par :

\begin{equation}
PD = N\left(\frac{\ln(D/V_0) - (\mu - \frac{\sigma^2}{2})T}{\sigma\sqrt{T}}\right)
\end{equation}

où $D$ représente la dette de l'entreprise, $V_0$ la valeur initiale des actifs, et $N(\cdot)$ la fonction de répartition de la loi normale standard.

\subsection{Score Z d'Altman modifié}

La version étendue du modèle d'Altman pour les entreprises non cotées s'exprime :

\begin{equation}
Z'' = 6.56X_1 + 3.26X_2 + 6.72X_3 + 1.05X_4
\end{equation}

où :
\begin{itemize}
\item $X_1 = \frac{\text{Fonds de roulement}}{\text{Total des actifs}}$
\item $X_2 = \frac{\text{Bénéfices non distribués}}{\text{Total des actifs}}$
\item $X_3 = \frac{\text{Résultat avant intérêts et impôts}}{\text{Total des actifs}}$
\item $X_4 = \frac{\text{Valeur comptable des capitaux propres}}{\text{Total des dettes}}$
\end{itemize}

Les seuils de classification sont : $Z'' > 2.6$ (zone sûre), $1.1 < Z'' < 2.6$ (zone grise), $Z'' < 1.1$ (zone de détresse).

\section{Annexe B : Algorithmes d'apprentissage automatique}

\subsection{Algorithme XGBoost}

L'objectif d'optimisation de XGBoost pour l'itération $t$ est :

\begin{equation}
\mathcal{L}^{(t)} = \sum_{i=1}^n l(y_i, \hat{y}_i^{(t-1)} + f_t(x_i)) + \Omega(f_t)
\end{equation}

où $l$ est la fonction de perte, $\hat{y}_i^{(t-1)}$ la prédiction à l'itération précédente, $f_t$ le nouvel arbre, et $\Omega(f_t)$ le terme de régularisation :

\begin{equation}
\Omega(f_t) = \gamma T + \frac{1}{2}\lambda \sum_{j=1}^T w_j^2
\end{equation}

avec $T$ le nombre de feuilles, $w_j$ le poids de la feuille $j$, $\gamma$ et $\lambda$ les paramètres de régularisation.

\subsection{Architecture LSTM}

Les équations gouvernant une cellule LSTM sont :

\begin{align}
f_t &= \sigma(W_f \cdot [h_{t-1}, x_t] + b_f) \\
i_t &= \sigma(W_i \cdot [h_{t-1}, x_t] + b_i) \\
\tilde{C}_t &= \tanh(W_C \cdot [h_{t-1}, x_t] + b_C) \\
C_t &= f_t * C_{t-1} + i_t * \tilde{C}_t \\
o_t &= \sigma(W_o \cdot [h_{t-1}, x_t] + b_o) \\
h_t &= o_t * \tanh(C_t)
\end{align}

où $f_t$, $i_t$, et $o_t$ sont respectivement les portes d'oubli, d'entrée et de sortie.

\section{Annexe C : Métriques d'évaluation détaillées}

\subsection{Courbe ROC et AUC}

L'aire sous la courbe ROC (AUC) se calcule par intégration :

\begin{equation}
AUC = \int_0^1 TPR(FPR^{-1}(x)) dx
\end{equation}

où $TPR = \frac{TP}{TP + FN}$ et $FPR = \frac{FP}{FP + TN}$.

\subsection{Coefficient de Gini}

Le coefficient de Gini, utilisé en scoring de crédit, est relié à l'AUC par :

\begin{equation}
\text{Gini} = 2 \times AUC - 1
\end{equation}

Un modèle parfait a un Gini de 1, tandis qu'un modèle aléatoire a un Gini de 0.

\subsection{Test de Kolmogorov-Smirnov}

La statistique KS mesure la séparation maximale entre les distributions des bons et mauvais payeurs :

\begin{equation}
KS = \max_s |F_1(s) - F_0(s)|
\end{equation}

où $F_1$ et $F_0$ sont les fonctions de répartition empiriques des classes positive et négative.

\section{Annexe D : Critères ESG détaillés}

\subsection{Critères environnementaux (E)}

Les indicateurs environnementaux comprennent :

\textbf{Émissions de carbone :}
\begin{itemize}
\item Scope 1 : Émissions directes des activités de l'entreprise
\item Scope 2 : Émissions indirectes liées à l'énergie consommée
\item Scope 3 : Autres émissions indirectes de la chaîne de valeur
\end{itemize}

\textbf{Gestion des ressources :}
\begin{itemize}
\item Consommation d'eau et efficacité hydrique
\item Gestion des déchets et taux de recyclage
\item Utilisation d'énergies renouvelables
\item Impact sur la biodiversité
\end{itemize}

\subsection{Critères sociaux (S)}

\textbf{Capital humain :}
\begin{itemize}
\item Diversité et inclusion (ratio hommes/femmes, représentation ethnique)
\item Formation et développement des employés
\item Santé et sécurité au travail (taux d'accidents)
\item Rémunération équitable et avantages sociaux
\end{itemize}

\textbf{Relations avec les parties prenantes :}
\begin{itemize}
\item Satisfaction client et qualité des produits
\item Impact sur les communautés locales
\item Chaîne d'approvisionnement responsable
\item Respect des droits humains
\end{itemize}

\subsection{Critères de gouvernance (G)}

\textbf{Structure de gouvernance :}
\begin{itemize}
\item Indépendance du conseil d'administration
\item Diversité du conseil (âge, genre, compétences)
\item Séparation des fonctions PDG/Président
\item Comités spécialisés (audit, rémunération, nomination)
\end{itemize}

\textbf{Éthique et transparence :}
\begin{itemize}
\item Politiques anti-corruption
\item Transparence fiscale
\item Protection des actionnaires minoritaires
\item Qualité du reporting financier
\end{itemize}

\section{Annexe E : Paramètres des modèles}

\subsection{Hyperparamètres XGBoost optimisés}

\begin{table}[h]
\centering
\begin{tabular}{|l|l|l|}
\hline
\textbf{Paramètre} & \textbf{Valeur optimale} & \textbf{Plage testée} \\
\hline
n\_estimators & 500 & [100, 1000] \\
max\_depth & 6 & [3, 10] \\
learning\_rate & 0.1 & [0.01, 0.3] \\
subsample & 0.8 & [0.6, 1.0] \\
colsample\_bytree & 0.8 & [0.6, 1.0] \\
gamma & 0.1 & [0, 0.5] \\
reg\_alpha & 0.01 & [0, 0.1] \\
reg\_lambda & 1.0 & [0.1, 10] \\
\hline
\end{tabular}
\caption{Hyperparamètres optimisés pour XGBoost}
\end{table}

\subsection{Architecture du réseau LSTM}

\begin{table}[h]
\centering
\begin{tabular}{|l|l|}
\hline
\textbf{Couche} & \textbf{Configuration} \\
\hline
LSTM 1 & 128 unités, return\_sequences=True \\
Dropout & Taux = 0.2 \\
LSTM 2 & 64 unités, return\_sequences=False \\
Dropout & Taux = 0.2 \\
Dense 1 & 32 unités, activation=ReLU \\
Dense 2 & 1 unité, activation=Sigmoid \\
\hline
\end{tabular}
\caption{Architecture du réseau LSTM}
\end{table}

\section{Annexe F : Pseudo-code des algorithmes principaux}

\subsection{Préprocessing des données ESG}

\begin{verbatim}
ALGORITHME PreprocessingESG
ENTRÉE: données_esg, données_financières
SORTIE: dataset_intégré

1. POUR chaque variable ESG v FAIRE
2.    Détecter les valeurs aberrantes (Q1 - 1.5*IQR, Q3 + 1.5*IQR)
3.    Imputer les valeurs manquantes par la médiane sectorielle
4.    Normaliser v selon Z-score : (v - μ) / σ
5. FIN POUR

6. Créer variables d'interaction ESG × Financières
7. Sélectionner variables par importance (Random Forest)
8. RETOURNER dataset_intégré
\end{verbatim}

\subsection{Validation croisée temporelle}

\begin{verbatim}
ALGORITHME ValidationCroiseeTemporelle
ENTRÉE: données, modèle, n_splits
SORTIE: scores_validation

1. Trier données par date
2. taille_split = len(données) / n_splits
3. POUR i = 1 à n_splits FAIRE
4.    début_train = 0
5.    fin_train = i * taille_split
6.    début_test = fin_train + 1
7.    fin_test = (i + 1) * taille_split
8.    
9.    Entraîner modèle sur [début_train:fin_train]
10.   Évaluer modèle sur [début_test:fin_test]
11.   Stocker score dans scores_validation[i]
12. FIN POUR
13. RETOURNER scores_validation
\end{verbatim}

\section{Annexe G : Tests statistiques complémentaires}

\subsection{Test de significativité des coefficients}

Pour les modèles logistiques, la statistique de Wald teste $H_0: \beta_i = 0$ :

\begin{equation}
W_i = \frac{\hat{\beta}_i^2}{Var(\hat{\beta}_i)} \sim \chi^2(1)
\end{equation}

\subsection{Test de stabilité des modèles (PSI)}

L'indice de stabilité de population (PSI) compare les distributions entre échantillons :

\begin{equation}
PSI = \sum_{i=1}^{10} (P_{dev,i} - P_{val,i}) \times \ln\left(\frac{P_{dev,i}}{P_{val,i}}\right)
\end{equation}

où $P_{dev,i}$ et $P_{val,i}$ sont les proportions dans le décile $i$ pour les échantillons de développement et validation.

Interprétation : PSI < 0.1 (stable), 0.1 ≤ PSI < 0.25 (changement modéré), PSI ≥ 0.25 (changement significatif).

\section{Annexe H : Glossaire des termes techniques}

\textbf{AUC (Area Under Curve)} : Aire sous la courbe ROC, mesure de performance pour les modèles de classification binaire.

\textbf{Backtesting} : Validation d'un modèle sur des données historiques pour évaluer sa performance prédictive.

\textbf{Bootstrapping} : Méthode de rééchantillonnage avec remise pour estimer la distribution d'une statistique.

\textbf{Credit Spread} : Différence entre le rendement d'une obligation corporate et d'une obligation sans risque de même maturité.

\textbf{Default Rate} : Taux de défaut, proportion d'emprunteurs qui ne remboursent pas leurs obligations.

\textbf{Expected Loss} : Perte attendue, calculée comme PD × LGD × EAD.

\textbf{Feature Engineering} : Processus de création et sélection de variables pertinentes pour l'apprentissage automatique.

\textbf{Gradient Boosting} : Méthode d'ensemble qui combine séquentiellement des modèles faibles pour corriger les erreurs.

\textbf{Loss Given Default (LGD)} : Taux de perte en cas de défaut, proportion de l'exposition perdue lors d'un défaut.

\textbf{Probability of Default (PD)} : Probabilité qu'un emprunteur ne puisse honorer ses obligations sur un horizon donné.

\textbf{Recall} : Sensibilité, proportion de vrais positifs correctement identifiés par le modèle.

\textbf{SHAP (SHapley Additive exPlanations)} : Méthode d'explication des prédictions basée sur la théorie des jeux coopératifs.

\section{Annexe I : Sources de données ESG}

\subsection{Principales agences de notation ESG}

\textbf{MSCI ESG Research} : Couvre plus de 8 500 entreprises avec une méthodologie basée sur l'exposition aux risques ESG et la gestion de ces risques.

\textbf{Sustainalytics} : Évalue les risques ESG non gérés à travers un score de 0 (risque négligeable) à 100+ (risque sévère).

\textbf{Refinitiv (ex-Thomson Reuters)} : Fournit des scores ESG basés sur plus de 630 métriques individuelles.

\textbf{S\&P Global ESG Scores} : Évalue la durabilité des entreprises selon une échelle de 0 à 100.

\subsection{Divergences entre agences}

Les corrélations entre scores ESG varient significativement :
\begin{itemize}
\item MSCI vs Sustainalytics : ρ ≈ 0.38
\item MSCI vs Refinitiv : ρ ≈ 0.54
\item Sustainalytics vs Refinitiv : ρ ≈ 0.42
\end{itemize}

Ces divergences s'expliquent par des différences de méthodologie, de pondération des critères, et de périmètre d'évaluation.

\section{Annexe J : Réglementation et standards}

\subsection{Taxonomie européenne}

La taxonomie européenne définit six objectifs environnementaux :
\begin{enumerate}
\item Atténuation du changement climatique
\item Adaptation au changement climatique
\item Utilisation durable et protection des ressources aquatiques et marines
\item Transition vers une économie circulaire
\item Prévention et réduction de la pollution
\item Protection et restauration de la biodiversité et des écosystèmes
\end{enumerate}

\subsection{Standards de reporting}

\textbf{TCFD (Task Force on Climate-related Financial Disclosures)} : Recommandations structurées autour de quatre piliers :
\begin{itemize}
\item Gouvernance
\item Stratégie
\item Gestion des risques
\item Métriques et objectifs
\end{itemize}

\textbf{SASB (Sustainability Accounting Standards Board)} : Standards sectoriels pour le reporting de durabilité financièrement matériel.

\textbf{GRI (Global Reporting Initiative)} : Standards mondiaux pour le reporting de durabilité couvrant les impacts économiques, environnementaux et sociaux.

\section{Annexe K : Limites et biais identifiés}

\subsection{Biais dans les données ESG}

\textbf{Biais de sélection} : Les entreprises qui reportent volontairement leurs données ESG peuvent être intrinsèquement différentes de celles qui ne le font pas.

\textbf{Biais de mesure} : Les métriques ESG peuvent ne pas capturer fidèlement les performances réelles en matière de durabilité.

\textbf{Biais temporel} : L'impact des initiatives ESG peut se manifester avec retard, créant un décalage entre les scores et les performances financières.

\subsection{Limites méthodologiques}

\textbf{Multicolinéarité} : Forte corrélation entre certaines variables ESG pouvant affecter la stabilité des modèles.

\textbf{Non-linéarité} : Relations complexes entre facteurs ESG et risque de crédit non toujours capturées par les modèles linéaires.

\textbf{Hétérogénéité sectorielle} : L'importance relative des facteurs ESG varie significativement entre secteurs d'activité.

\section{Annexe L : Perspectives de recherche future}

\subsection{Intégration de données alternatives}

\textbf{Données satellitaires} : Mesure directe des émissions, déforestation, activité industrielle.

\textbf{Analyse textuelle} : Traitement automatique des rapports annuels, communiqués de presse, médias sociaux.

\textbf{Données comportementales} : Analyse des patterns de comportement des dirigeants et employés.

\subsection{Modèles avancés}

\textbf{Graph Neural Networks} : Modélisation des relations complexes entre entreprises, fournisseurs, et parties prenantes.

\textbf{Transformer Models} : Application des architectures attention pour l'analyse de séries temporelles financières.

\textbf{Causal Inference} : Identification des relations causales entre pratiques ESG et performance financière.

\subsection{Défis émergents}

\textbf{Greenwashing automatique} : Développement d'algorithmes pour détecter les pratiques de greenwashing.

\textbf{Risques climatiques physiques} : Intégration des modèles climatiques dans l'évaluation du risque de crédit.

\textbf{Transition énergétique} : Modélisation des risques associés à la transition vers une économie bas-carbone.